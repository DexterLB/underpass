\documentclass[main.tex]{subfiles}
\begin{document}

\section{Introduction}
This project seeks to provide means for translating from a subset of English
into Overpass queries using a Combinatory Categorial Grammar.

\subsection{Motivation}
It is often the case that humans need to communicate a concrete question
or order (a query) to a computer. The most straightforward way for this to happen is
to have the human write out their query in a domain-specific language (DSL),
specially crafted to handle the given type of query. This approach, albeit
very efficient when done correctly, requires substantial upfront effort:
the human needs to get familiar with the DSL and must have access to a specific
medium\marginpar{Искаш да кажеш посредник ли? Може би ``means'' или ``intermediary'', ``medium'' е по-скоро ``storage medium''.}
(usually a keyboard and a screen). Thus, in some cases it is desirable
to form an expression in a natural language (or a language very similar to a
subset of a natural language) and have it translated to a query in the given
DSL.

The point is not to translate \emph{any} sentence in the source language
that has semantics applicable to the target DSL, but rather to define a
subset of the source language, in which a human could express a query
reasonably easily.

Most DSLs have compositional semantics, which makes them great candidates
for generation using Combinatory Category Grammars (CCGs) \cite[p.~181]{nts}.

One such DSL is the Overpass language, used by the open source project
OpenStreetMap for making queries to the map database. It has been chosen as
a concrete target language for this project in order to assess the usefulness
of CCG for translating from a natural language subset to a compositional
DSL.
\end{document}
