\documentclass[main.tex]{subfiles}
\begin{document}
\section{$\beta$-reduction and (un)currying}
\label{betacurry}

\subsubsection{$\beta$-reduction}
This section briefly explains the concept of $\beta$-reduction. For more details,
see \cite{pierce}.

\begin{defn}
    For $M, N \in \plambda{T}{C}$, the substitution of the free variable
    $x$ within $M$ with $N$ is defined inductively as follows:
    \[
        \subst{M}{x}{N} \defeq
        \begin{cases*}
            N, & $M = x$ \\
            y, & $M = y \in \lvars \setminus \{ x \}$ \\
            \lambda x : \sigma \abstr \subst{P}{x}{N},
                & $M = (\lambda x : \sigma \abstr P), y \ne x \land y \not\in \fv{P}$ \\
            \subst{A}{x}{N} \subst{B}{x}{N}, & M = (AB) \\
        \end{cases*}
    \]

    Note that this function is partial, but this can be swept under the rug
    by renaming bound variables when needed \cite{pierce}.
\end{defn}

We will now define the $\beta$-reduction relation $\betared$ between $\lambda$-terms:
\begin{defn}
    Two terms are in $\betared$ relation only in the following case:
    \[ (\lambda x : \sigma \abstr M) N \betared \subst{M}{x}{N} \]

    If $A \betared B$, $A$ is called a \emph{$\beta$-redex}.
\end{defn}

The relation $\bbetared$ $\beta$-reduces any subterm:
\begin{defn}
    The $\bbetared$ relation is defined inductively:
    \begin{itemize}
        \item $A \betared B \implies A \bbetared B$
        \item $A \bbetared B \implies AP \bbetared BP$
        \item $A \bbetared B \implies PA \bbetared PB$
        \item $A \bbetared B \implies (\lambda x : \sigma \abstr A)
            \bbetared (\lambda x : \sigma \abstr B)$
    \end{itemize}
\end{defn}

\begin{defn}
    A term $M \in \plambda{T}{C}$ is called \emph{fully $\beta$-reduced} if:
    \[ \forall N \in \plambda{T}{C} (\neg (M \betared N)) \]
\end{defn}

\subsubsection{Uncurrying}
We can represent every $\lambda$-term as a function (\emph{head}) applied
to a sequence of arguments:
\begin{itemize}
    \item if the given term $M$ is not an application, its head is $M$ itself
        with no arguments.
    \item if the given term is $M = (AB)$, its head is the head of $A$ and its
        arguments are $B$ appended to the arguments of $A$.
\end{itemize}

\begin{example}
    The term $P (Q R) (\lambda x: \sigma \abstr S) T$ has head $P$ and
    arguments $(Q R)$, $(\lambda x: \sigma \abstr S)$ and $T$.
\end{example}
\end{document}
