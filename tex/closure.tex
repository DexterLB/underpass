\documentclass[main.tex]{subfiles}
\begin{document}

The original purpose of this thesis was to create an end-to-end
solution for translating some subset of English into Overpass.
In the process it became clear that defining the exact subset of English
was not as interesting as expected, so the project grew to allow
definition of frontends for any natural language subsets. This shifted the
focus into the design of a rule definition language, which felt more
fulfilling.

It seems clear that this CCG-based approach can't model the input natural
language as well as some methods based on statistics would have
(the language subset modelled by CCG feels more formal than actual
natural language sentences), but it
may prove useful for many scenarios: this ``formalness'' ensures predictability
of the system and still models enough of the natural language for queries
to be semantically-recognisable to non-experts. Interfacing such a system
may be much easier to learn than working directly with a formal language
such as Overpass.

In retrospect, the subjectively most interesting part of this thesis was probably
incorporating the concept of subtyping into CCGs and using it to model
real-world semantics, and also the most tedious, since the details arising
from contravariance had to be considered in most prooves.

Even without using Overpass as a target formal language, this project might be
useful for generating queries into other formal languages by simply
substituting the backend.

\subsection{Future work}
The current implementation is fairly basic and may be extended along multiple
axes:

\begin{itemize}
    \item Implement more Overpass features into the translator to cover a greater
        target domain
    \item Allow assigning weights to CCG rules in order to be able to prioritise
        different parses for the same query
    \item Implement a version of the Vijay-Shanker algorithm for CCG parsing
        in order to improve performance
    \item Avoid $\beta$-reduction on Minipass terms of type $GSet$ in order
        to eliminate some duplicate subqueries
    \item Use Eisner normal form parsing to handle ambiguity better
    \item Use learning methods to automatically extend grammars by using
        query corpora
    \item ...
\end{itemize}

\pagebreak
\subsection{Conclusion}
Mostly harmless.
\end{document}
