\documentclass[main.tex]{subfiles}
\begin{document}

\subsection{Simply typed $\lambda$-calculus}

For representing compositional terms, a version of simply typed lambda calculus
will be used.

\subsubsection{Types}
\begin{defn}
    A set $T$ is a \emph{meet-semilattice with top and bottom} when:
    \begin{itemize}
        \item the partial order relation
            $\less \subseteq T \times T$ is defined
        \item the operator $\meet : T \times T \rightarrow T$ is defined
        \item $\top \in T, \bot \in T$
        \item For any $\sigma, \tau, \nu \in T$:
            \begin{itemize}
                \item $(\sigma \meet \tau) \meet \nu = \sigma \meet (\tau \meet \nu)$
                \item $\sigma \meet \tau = \tau \meet \sigma$
                \item $\sigma \meet \tau = \nu \implies \nu \less \sigma$
                \item $\sigma \meet \tau = \nu \implies \nu \less \tau$
                \item $\sigma \meet \sigma = \sigma$
                \item $\sigma \meet \top = \top \meet \sigma = \sigma$
                \item $\sigma \meet \bot = \bot \meet \sigma = \bot$
                \item $\sigma \less \top$, $\bot \less \sigma$
            \end{itemize}
    \end{itemize}

\end{defn}

\begin{defn}
    A set $T$ is a \emph{type semilattice} when:
    \begin{itemize}
        \item For any $\sigma, \tau \in T$, $(\sigma \tot \tau) \in T$
        \item $T$ is a bounded meet-semilattice with top and bottom.
    \end{itemize}

    For convenience, the $\rightarrow$ operator will be regarded as
    right-associative.
\end{defn}

\begin{defn}
    For a meet-semilattice with top and bottom $T$, we can define the
    \emph{type closure} $\mathcal{T}(T)$ and extend $\less$ and $\meet$ as follows:
    \begin{itemize}
        \item $\sigma \in T \implies \sigma \in \mathcal{T}(T)$
        \item $(\sigma' \rightarrow \tau'), (\sigma'' \rightarrow \tau'') \in
            \mathcal{T}(T) \implies
            (\sigma' \rightarrow \tau') \less (\sigma'' \rightarrow \tau'')
            \Leftrightarrow
            (\sigma' \less \sigma'') \& (\tau' \less \tau'')$
        \item $(\sigma' \rightarrow \tau'), (\sigma'' \rightarrow \tau'') \in
            \mathcal{T}(T) \implies
            (\sigma' \rightarrow \tau') \meet (\sigma'' \rightarrow \tau'')
            =
            (\sigma' \meet \sigma'') \rightarrow (\tau' \meet \tau'')$
        \item $(\sigma' \rightarrow \sigma'') \in \mathcal{T}(T), \tau \in T
            \implies (\sigma' \rightarrow \sigma'') \meet \tau
            = \tau \meet (\sigma' \rightarrow \sigma'') = \bot$
    \end{itemize}
\end{defn}
\begin{prop}
    If $T$ is a meet-semilattice with top and bottom, $\mathcal{T}(T)$ is
    a type semilattice.
\end{prop}

\begin{defn}
    We call a set $X$ \emph{typed in} $T$ $\iff$ there is a function
    \[ typeof : X \rightarrow T \] and $T$ is a type semilattice.
\end{defn}

\subsubsection{Terms}
\begin{defn}
    Let $\mathbb{V}$ be an infinite countable set. We shall call its elements
    \emph{variables}.
\end{defn}

\begin{defn}
    Let $C$ be a countable set, typed in $T$,
    whose elements we call \emph{constants}.

    The set $\Lambda_T^C$ is defined inductively:
    \begin{itemize}
        \item for $c \in C, c \in \Lambda_T^C$
            and $typeof(c)$ is already defined.

    \end{itemize}
\end{defn}

\end{document}
