\documentclass[main.tex]{subfiles}
\begin{document}

\subsection{Pure CCG}
\label{sec:ccg}
The CCG formalism is used in a multitude of variants. The form presented here
will be most basic and of little practical use: it will only allow composition
rules and the only restriction will be a global maximum arity.

It is based on the Vijay-Shanker CCG formalism \cite{shanker}.

This is enough for understanding the parsing algorithms, and can be further
extended to include features such as type-raising,
arbitrary rule restrictions, slash modalities,
category features, category variables and feature variables.

Let $Cat$ be a magical set whose elements we call
\emph{atomic categories}.

\begin{defn}
    For a set of atomic categories $\tau$, its
    \emph{categorial closure} $C(\tau)$ is defined as follows:
    \begin{enumerate}
        \item \label{itm:atomic} $A \in \tau \Rightarrow A \in C(\tau)$
        \item \label{itm:right}  $X, Y \in C(\tau) \Rightarrow \lp X \rc Y \rp \in C(\tau)$
        \item \label{itm:left}   $X, Y \in C(\tau) \Rightarrow \lp X \lc Y \rp \in C(\tau)$
    \end{enumerate}

    Letters like $A, B, C$ will be used to denote atomic categories,
    while letters like $X, Y, Z$ will be used to
    denote complex categories (produced by rules \ref{itm:left} and \ref{itm:right}).

    Such expressions are called \emph{categories}. The $\mc$ symbol will be
    used to denote any slash (when the distinction between $\lc$ and $\rc$
    does not matter). Categories will also be considered left-associative.
    Thus we can write $X_1 \mc X_2 \mc X_3 \mc X_4$ to denote
    $\lp \lp \lp X_1 \mc X_2 \rp \mc X_3 \rp \mc X_4 \rp$. It's useful to note that $X_1$ may
    be decomposed into its constituents until it becomes an atomic category:
    this means that every category can be written in the form
    $A \mc X_1 \mc X_2 \mc ... \mc X_n$, where $A$ is atomic. We say that
    $n$ is the category's \emph{arity}. $A$ is called its \emph{target} or
    return type, while
    $X_1 ... X_n$ are called its \emph{arguments} or argument types.

    For any string $\alpha \in (\lb C(\tau) \rb \cup \Sigma)^*$, where $\Sigma$ is a
    finite set, we define $C(\alpha)$ to be the set of all categories which
    can be found in $\alpha$, namely:
    \begin{itemize}
        \item $C(\varepsilon) = \emptyset$
        \item for $a \in \Sigma$, $C(a\beta) = C(\beta)$
        \item for $X \in C(\tau)$, $C(\lb X \rb \beta) = X \cup C(\beta)$
    \end{itemize}
\end{defn}

\begin{defn}
    Concatenating categories: If $X \in C(\tau)$,
    $Y = A \mci{1} Y_1 \mci{2} Y_2 ... \mci{m} Y_m \in C(\tau)$, we define
    \[ X \circ_{\rc} Y = X \rc A \mci{1} Y_1 \mci{2} Y_2 ... \mci{m} Y_m \]
    \[ X \circ_{\lc} Y = X \lc A \mci{1} Y_1 \mci{2} Y_2 ... \mci{m} Y_m \]

    Since using this notation everywhere is tedious, $X \mc Y$ shall mean
    $X \circ_{\mc} Y$ throughout this text, in contrast to
    $X \mc \lp Y \rp$.
\end{defn}

\begin{defn}
    Length of category: for $X = A \mc X_1 \mc X_2 \mc ... \mc X_n \in C(\tau)$,
    $|X| = n$.
\end{defn}

\begin{defn}
    $ G = \langle \Sigma, N, S, f, n \rangle $ is a \emph{Combinatory Categorial Grammar}, where
    \begin{itemize}
        \item $ \Sigma $ is the (finite) set of \emph{terminals}
        \item $ N $ is the (finite) set of \emph{non-terminals} (atomic categories)
        \item $ S \in N $ is the \emph{target category}
        \item $ f : \Sigma \rightarrow \hat{N} $, where $\hat{N}$ is the set of
            \textbf{finite} subsets of $C(N)$, is the function for interpreting
            terminals
        \item $ n \in \mathbb{N} $ is the \emph{maximum composition arity}
    \end{itemize}
\end{defn}

\subsubsection{Derivations}

\begin{defn}
    For a CCG $G$, we can construct a set $R \subset (\lb C(N) \rb) \times (\Sigma \cup \lb C(N) \rb)^*$
    of \emph{rule instances}. Instead of $(\alpha, \beta) \in R$, we will write
    $\alpha \rightarrow \beta$.

    \begin{itemize}
        \item If $ a \in \Sigma, X \in f(a) $, then \[ \lb X \rb \rightarrow a \]
        \item If $ X \rc Y \in C(N), Y \mci{1} Z_1 \mci{2} Z_2 ... \mci{m} Z_m \in C(N), 0 \leq m \leq n $
            then \[ \lb X \mci{1} Z_1 \mci{2} Z_2 ... \mci{m} Z_m \rb \rightarrow \lb X \rc Y \rb \lb Y \mci{1} Z_1 \mci{2} Z_2 ... \mci{m} Z_m \rb \]

            Furthermore, we call $ X \rc Y $ the \emph{primary category} of the rule
            instance, while $ Y \mci{1} Z_1 \mci{2} Z_2 ... \mci{m} Z_m $ is its
            \emph{secondary category}.
            We also call $ X \rc Y $ the \emph{left category} of the rule instance,
            and $  Y \mci{1} Z_1 \mci{2} Z_2 ... \mci{m} Z_m $ its \emph{right category}.
        \item If $ X \lc Y \in C(N), Y \mci{1} Z_1 \mci{2} Z_2 ... \mci{m} Z_m \in C(N), 0 \leq m \leq n $
            then \[ \lb X \mci{1} Z_1 \mci{2} Z_2 ... \mci{m} Z_m \rb \rightarrow \lb Y \mci{1} Z_1 \mci{2} Z_2 ... \mci{m} Z_m \rb \lb X \lc Y \rb \]
            Here, $ X \lc Y $ is the \emph{primary} and \emph{right} category,
            while $ Y \mci{1} Z_1 \mci{2} Z_2 ... \mci{m} Z_m $ is the \emph{secondary}
            and \emph{left} category.
    \end{itemize}
\end{defn}

\begin{defn}
    Derivation process
    \begin{itemize}
        \item If $\beta \rightarrow \beta'$, then $\alpha \beta \gamma \Rightarrow \alpha \beta' \gamma$
        \item Let $\Rightarrow^*$ be the reflexive and transitive closure of $\Rightarrow$.
        \item If $\alpha \Rightarrow^* \beta$, we can write
            \[ \mu: \alpha \Rightarrow \alpha_1 \Rightarrow ... \Rightarrow \alpha_r = \beta \]
            and call $\mu$ a \emph{derivation} for $G$. Then $C(\mu) = \bigcup\limits_{i=1}^{r} C(\alpha_i)$
            is the set of all categories used in $\mu$.
        \item $L(G) := \{ \alpha \in \Sigma^* \mid \lb S \rb \Rightarrow^* \alpha \}$
    \end{itemize}
\end{defn}

\begin{prop}\label{prop:cfg}
    A CCG $G = \langle \Sigma, N, S, f, n \rangle$ is equivalent to an
    (albeit infinite) context-free grammar
    $G^C = \langle \Sigma, \lb C(N) \rb, R, \lb S \rb \rangle$.

    Moreover, let $\mu: \alpha_1 \Rightarrow ... \Rightarrow \alpha_r$
    be a derivation for $G$.
    We can construct a finite context-free grammar
    $G_\mu^C = \langle \Sigma, \lb C(\mu) \rb, \restr{R}{\lb C(\mu) \rb}, \lb S \rb \rangle$
    which produces $\mu$, and whose derivations are also derivations in $G$.
\end{prop}
\begin{proof}
    This follows because our definition of $\Rightarrow$ for $G$ and the context-free
    definition of $\Rightarrow$ for $G^C$ are the same.
\end{proof}

As per the given construction, there is no way for a CCG to generate the empty
string.
\begin{property}
    For any CCG $G$, $\varepsilon \notin L(G)$
\end{property}

\begin{prop}\label{prop:concat}
    Concatenativity of CCG derivation

    For $\alpha \neq \varepsilon, \beta \neq \varepsilon$, the following two are equivalent:
    \begin{itemize}

        \item $\exists \alpha' \neq \varepsilon, \beta' \neq \varepsilon$
            such that $\gamma = \alpha' \beta', \alpha \Rightarrow^* \alpha', \beta \Rightarrow^* \beta'$
        \item $\alpha \beta \Rightarrow^* \gamma$
    \end{itemize}
\end{prop}
\begin{proof}
    We can use Proposition \ref{prop:cfg}:

    \begin{itemize}
        \item if $\alpha \Rightarrow^* \alpha'$ and $\beta \Rightarrow^* \beta'$,
            then $\alpha \beta \Rightarrow^* \alpha' \beta \Rightarrow^* \alpha' \beta' = \gamma$.
        \item if $\alpha \beta \Rightarrow^* \gamma, \alpha \neq \varepsilon, \beta \neq \varepsilon$,
            we can fix $\mu: \alpha \beta \Rightarrow \gamma_1 \Rightarrow \gamma_2 ... \Rightarrow \gamma_t = \gamma$.
            Then $\mu$ is a valid derivation in the context-free grammar $G^C_{\mu}$,
            which means $\alpha$ and $\beta$ generate separate derivation
            subtrees, thus $\exists \alpha' \beta' = \gamma$ such that
            $\alpha \Rightarrow^* \alpha'$ and $\beta \Rightarrow^* \beta'$,
            which also holds for $G$.
    \end{itemize}
\end{proof}

\begin{defn}
    Derivation tree

    Derivation trees for CCG are ordered binary trees with categories labeling
    their internal nodes and terminals labeling their leaves.

    We shall also consider the concept of \emph{primary edges} in the context
    of derivation trees, in correspondence to the concept of
    \emph{primary categories} in the context of derivations. Primary edges
    will be drawn with a thick line.

    We will construct the set $ \mathbb{T} $ of all derivation trees, as well
    as the functions $ crown: \mathbb{T} \rightarrow ( \lb C(N) \rb \cup \Sigma )^* $,
    $ root: \mathbb{T} \rightarrow C(N) \cup \Sigma$,
    $ primary : \mathbb{T} \rightarrow \mathbb{T} $ and
    $ prefix : \mathbb{T} \rightarrow C(N) \cup \Sigma $.

    If there exists $\tau \in \mathbb{T}$ such that $root(\tau) = X \in C(N),
    crown(\tau) = \alpha$, we denote $\lb X \rb \Rrightarrow \alpha$.

    Derivation trees are constructed as follows:

    \begin{itemize}
        \item If $ a \in \Sigma $, then
            $ a \in \mathbb{T}, crown(a) = a, root(a) = a, primary(a) = a, prefix(a) = a$.
        \item If $ X \in C(N) $ then $ X \in \mathbb{T}, crown(X) = X, primary(X) = X, prefix(X) = X$
        \item If $ a \in \Sigma, X \in f(a) $, then
            \begin{center}
                \tree{.$X$ \edge[very thick]; $a$ } ( $ \pi $ )
            \end{center}
            $ \pi \in \mathbb{T}, crown(\pi) = a, root(\pi) = X, primary(\pi) = a, prefix(\pi) = a $
        \item If
            \begin{center}
                \tree{.{$X \rc Y$} \edge[roof]; $\alpha$} ($\pi'$)
                \quad and \quad
                \tree{.{$Y \mci{1} Z_1 \mci{2} Z_2 ... \mci{m} Z_m$} \edge[roof]; $\beta$} ($\pi''$)
            \end{center}

            are derivation trees such that $ root(\pi') = X \rc Y $,
            $ root(\pi'') = Y \mci{1} Z_1 \mci{2} Z_2 ... \mci{m} Z_m $,
            $ crown(\pi') = \alpha, crown(\pi'') = \beta $
            where $m \leq n$, then
            \begin{center}
                \tree{.{$X \mci{1} Z_1 \mci{2} Z_2 ... \mci{m} Z_m$}
                    \edge[very thick];
                    [ .{$X \rc Y$} \edge[roof]; $\alpha$ ]
                    [ .{$Y \mci{1} Z_1 \mci{2} Z_2 ... \mci{m} Z_m$} \edge[roof]; $\beta$ ]
                } ($\tau$)
            \end{center}
            $ \tau \in \mathbb{T}, root(\tau) = X \mci{1} Z_1 \mci{2} Z_2 ... \mci{m} Z_m $,
            $ crown(\tau) = \alpha \beta $, $primary(\tau) = \pi'$, $prefix(\tau) = X$
        \item If
            \begin{center}
                \tree{.{$Y \mci{1} Z_1 \mci{2} Z_2 ... \mci{m} Z_m$} \edge[roof]; $\beta$} ($\pi'$)
                \quad and \quad
                \tree{.{$X \lc Y$} \edge[roof]; $\alpha$} ($\pi''$)
            \end{center}
            are derivation trees such that
            $ root(\pi') = Y \mci{1} Z_1 \mci{2} Z_2 ... \mci{m} Z_m $,
            $ root(\pi'') = X \lc Y $,
            $ crown(\pi') = \alpha, crown(\pi'') = \beta $
            where $m \leq n$, then
            \begin{center}
                \tree{.{$X \mci{1} Z_1 \mci{2} Z_2 ... \mci{m} Z_m$}
                    [ .{$Y \mci{1} Z_1 \mci{2} Z_2 ... \mci{m} Z_m$} \edge[roof]; $\alpha$ ]
                    \edge[very thick];
                    [ .{$X \rc Y$} \edge[roof]; $\beta$ ]
                } ($\tau$)
            \end{center}
            $ \tau \in \mathbb{T}, root(\tau) = X \mci{1} Z_1 \mci{2} Z_2 ... \mci{m} Z_m $,
            $ crown(\tau) = \alpha \beta $, $primary(\tau) = \pi''$, $prefix(\tau) = X$
    \end{itemize}

\end{defn}

\begin{prop}
    Equivalence of derivation trees and derivations

    $ \lb X \rb \Rightarrow^* \alpha \iff \lb X \rb \Rrightarrow \alpha $

    Note: this also extends for the case where we have a terminal on the left,
    but it is not worth mentioning.
\end{prop}
\begin{proof}
    First, let there exist a derivation tree for $ \lb X \rb \Rrightarrow \alpha $.
    $\lb X \rb \Rightarrow^* \alpha$ can be proven by induction over constructing
    the tree.
    \begin{itemize}
        \item For the trivial cases, the derivation has 0 steps.
        \item For a tree
            \centree{.$X$ \edge[very thick]; $a$ }
            and $X \in f(a)$, we have $\alpha = a, \lb X \rb \rightarrow a$
        \item For a tree
            \centree{.{$X \mci{1} Z_1 \mci{2} Z_2 ... \mci{m} Z_m$}
                \edge[very thick];
                [ .{$X \rc Y$} \edge[roof]; $\beta$ ]
                [ .{$Y \mci{1} Z_1 \mci{2} Z_2 ... \mci{m} Z_m$} \edge[roof]; $\gamma$ ]
            }
            we have $\alpha = \beta \gamma$, and inductively
            $\lb Y \mci{1} Z_1 \mci{2} Z_2 ... \mci{m} Z_m \rb \Rightarrow^* \beta$ and
            $\lb X \rc Y \rb \Rightarrow^* \alpha$. Thus,
            $\lb X \mci{1} Z_1 \mci{2} Z_2 ... \mci{m} Z_m \rb \Rightarrow^*$
            $\lb X \lc Y \rb \lb Y \mci{1} Z_1 \mci{2} Z_2 ... \mci{m} Z_m \rb$
            $ \Rightarrow^* \beta \gamma = \alpha$
        \item For a tree
            \centree{.{$X \mci{1} Z_1 \mci{2} Z_2 ... \mci{m} Z_m$}
                [ .{$Y \mci{1} Z_1 \mci{2} Z_2 ... \mci{m} Z_m$} \edge[roof]; $\beta$ ]
                \edge[very thick];
                [ .{$X \lc Y$} \edge[roof]; $\gamma$ ]
            }
            we have $\alpha = \beta \gamma$, and inductively
            $\lb Y \mci{1} Z_1 \mci{2} Z_2 ... \mci{m} Z_m \rb \Rightarrow^* \beta$ and
            $\lb X \lc Y \rb \Rightarrow^* \alpha$. Thus,
            $\lb X \mci{1} Z_1 \mci{2} Z_2 ... \mci{m} Z_m \rb \Rightarrow^*$
            $\lb Y \mci{1} Z_1 \mci{2} Z_2 ... \mci{m} Z_m \rb \lb X \lc Y \rb$
            $ \Rightarrow^* \beta \gamma = \alpha$
    \end{itemize}

    Second, we can prove the inverse implication (given a derivation,
    construct the derivation tree) by induction over derivation
    length. Let $\mu: \lb X \rb \Rightarrow \beta \Rightarrow^* \alpha$
    be a derivation for $G$.
    \begin{itemize}
        \item $\beta = \lb X \rb$. Then this is the trivial reflexive case
            and $\lb X \rb$ is a valid derivation tree in itself.
        \item $\beta = a \in \Sigma, X \in f(a)$. Then $\alpha = \beta$ and
            \centree{.$X$ \edge[very thick]; $a$ }
            is a derivation tree for $ \lb X \rb \Rrightarrow a = \alpha $
        \item $X = W \mci{1} Z_1 \mci{2} Z_2 ... \mci{m} Z_m$ and
            $\beta = \lb W \rc Y \rb \lb Y \mci{1} Z_1 \mci{2} Z_2 ... \mci{m} Z_m \rb$.
            According to Proposition \ref{prop:concat}, there exist $\eta, \xi$ such that:
            \begin{itemize}
                \item $\alpha = \eta \xi$
                \item $\lb W \rc Y \rb \Rightarrow^* \eta$
                \item $\lb Y \mci{1} Z_1 \mci{2} Z_2 ... \mci{m} Z_m \rb \Rightarrow^* \xi$
            \end{itemize}
            However, according to the inductive hypothesis, this means that we
            have the derivation trees
            \begin{center}
                \tree{.{$W \rc Y$} \edge[roof]; $\eta$}
                and
                \tree{.{$Y \mci{1} Z_1 \mci{2} Z_2 ... \mci{m} Z_m$} \edge[roof]; $\xi$}
            \end{center}
            Thus, by definition, we can construct the derivation tree
            \centree{.{$X = W \mci{1} Z_1 \mci{2} Z_2 ... \mci{m} Z_m$}
                \edge[very thick];
                [ .{$W \rc Y$} \edge[roof]; $\eta$ ]
                [ .{$Y \mci{1} Z_1 \mci{2} Z_2 ... \mci{m} Z_m$} \edge[roof]; $\xi$ ]
            }
            for $\lb X \rb \Rrightarrow \alpha$.
        \item $X = W \mci{1} Z_1 \mci{2} Z_2 ... \mci{m} Z_m$ and
            $\beta = \lb Y \mci{1} Z_1 \mci{2} Z_2 ... \mci{m} Z_m \rb \lb W \lc Y \rb$.
            According to Proposition \ref{prop:concat}, there exist $\eta, \xi$ such that:
            \begin{itemize}
                \item $\alpha = \eta \xi$
                \item $\lb Y \mci{1} Z_1 \mci{2} Z_2 ... \mci{m} Z_m \rb \Rightarrow^* \eta$
                \item $\lb W \lc Y \rb \Rightarrow^* \xi$
            \end{itemize}
            However, according to the inductive hypothesis, this means that we
            have the derivation trees
            \begin{center}
                \tree{.{$Y \mci{1} Z_1 \mci{2} Z_2 ... \mci{m} Z_m$} \edge[roof]; $\eta$}
                and
                \tree{.{$W \lc Y$} \edge[roof]; $\xi$}
            \end{center}
            Thus, by definition, we can construct the derivation tree
            \centree{.{$X = W \mci{1} Z_1 \mci{2} Z_2 ... \mci{m} Z_m$}
                [ .{$Y \mci{1} Z_1 \mci{2} Z_2 ... \mci{m} Z_m$} \edge[roof]; $\eta$ ]
                \edge[very thick];
                [ .{$W \lc Y$} \edge[roof]; $\xi$ ]
            }
            for $\lb X \rb \Rrightarrow \alpha$.
    \end{itemize}
\end{proof}

\begin{defn}
    A path in a derivation tree is a \emph{primary path} if all of its edges
    are primary.
\end{defn}

\begin{defn}
    A primary path in a derivation tree has a leaf $X \mc^* Z$ and prefix $X$
    if:
    \begin{itemize}
        \item $X \mc^* Z$ is a leaf
        \item the path from it to the root is primary
        \item every category in the primary path has the form $X \mc^* Y$
            for some $Y \in C(N)$.
    \end{itemize}
\end{defn}

\subsubsection{The Categorial CYK algorithm}

This is the most straight-forward algorithm for parsing a string with the help
of CCG. It is easy to understand, but has exponential runtime with respect to
the input word length.

It follows the same logic as the original CYK algorithm for context-free
grammars, with the difference that it uses the asymmetric combinator rules
from CCG to produce items.

Let $ G = \langle \Sigma, N, S, f, n \rangle $ be a CCG and $w = w_1 ... w_k$
be a word.

The algorithm recursively builds a set $P$ of \emph{items} in the form
$(X, i, j)$, $X \in C(N), 1 \leq i \leq j \leq k$ and aims to produce the item
$(S, 1, k)$.

\begin{enumerate}
\label{cyk:rules}
    \item If $X \in f(w_i)$, then $(X, i, i) \in P$
    \item If $(X \rc Y, i, p) \in P, (Y \mci{1} Z_1 \mci{2} Z_2 ... \mci{m} Z_m, p + 1, j) \in P$,
        then $(X \mci{1} Z_1 \mci{2} Z_2 ... \mci{m} Z_m, i, j) \in P$
    \item If $(X \lc Y, p + 1, j) \in P, (Y \mci{1} Z_1 \mci{2} Z_2 ... \mci{m} Z_m, i, p) \in P$,
        then $(X \mci{1} Z_1 \mci{2} Z_2 ... \mci{m} Z_m, i, j) \in P$
\end{enumerate}

To reason about the algorithm, we will use the following invariant:
\begin{equation}
    \lb X \rb \Rightarrow^* w_i ... w_j \iff (X, i, j) \in P
\end{equation}

\begin{prop}
    The algorithm is \textbf{correct}: if $(S, 1, k) \in P$, then $w \in L(G)$.
\end{prop}
\begin{proof}
    We have to prove the right-to-left direction of the invariant:
    \begin{equation}\label{eq:left}
        (X, i, j) \in P \implies \lb X \rb \Rightarrow^* w_i ... w_j
    \end{equation}

    From which would follow $ (S, 1, k) \in P \implies \lb S \rb \Rightarrow^* w_1 ... w_k$,
    which is what we need to prove.

    So, let $(X, i, j) \in P$ and inductively suppose that (\ref{eq:left}) is true
    for any shorter substrings of $w$. There are 3 rules by which this item has appeared:
    \begin{enumerate}
        \item $i = j, X \in f(w_i)$: This means
            $\lb X \rb \rightarrow w_i \implies \lb X \rb \Rightarrow^* w_i = w_i ... w_j$
        \item \label{itm:rightslash} $i < j$ and
            \begin{itemize}
                \item $X = W \mci{1} Z_1 \mci{2} Z_2 ... \mci{m} Z_m$
                \item $(W \rc Y, i, p) \in P$
                \item $(Y \mci{1} Z_1 \mci{2} Z_2 ... \mci{m} Z_m, p + 1, j) \in P$
            \end{itemize}
            for some $i \leq p < j, m \leq n$. By inductive hypothesis,
            $\lb W \rc Y \rb \Rightarrow^* w_i ... w_p$
            and
            $\lb Y \mci{1} Z_1 \mci{2} Z_2 ... \mci{m} Z_m \rb \Rightarrow^* w_{p + 1} ... w_j$.
            Then, by the definition of $\rightarrow$ and by concatenativity of
            $\Rightarrow$, we have:
            \[
                \lb X \rb = \lb W \mci{1} Z_1 \mci{2} Z_2 ... \mci{m} Z_m \rb
                \rightarrow
                \lb W \rc Y \rb \lb Y \mci{1} Z_1 \mci{2} Z_2 ... \mci{m} Z_m \rb
                \Rightarrow^*
                w_i ... w_p w_{p + 1} ... w_j = w
            \]
        \item $i < j$ and (analogous to \ref{itm:rightslash})
            \begin{itemize}
                \item $X = W \mci{1} Z_1 \mci{2} Z_2 ... \mci{m} Z_m$
                \item $(W \lc Y, p + 1, j) \in P$
                \item $(Y \mci{1} Z_1 \mci{2} Z_2 ... \mci{m} Z_m, i, p) \in P$
            \end{itemize}
            for some $i \leq p < j, m \leq n$. By inductive hypothesis,
            $\lb W \rc Y \rb \Rightarrow^* w_{p + 1} ... w_j$
            and
            $\lb Y \mci{1} Z_1 \mci{2} Z_2 ... \mci{m} Z_m \rb \Rightarrow^* w_i ... w_p$.
            Then, by the definition of $\rightarrow$ and by concatenativity of
            $\Rightarrow$, we have:
            \[
                \lb X \rb = \lb W \mci{1} Z_1 \mci{2} Z_2 ... \mci{m} Z_m \rb
                \rightarrow
                \lb Y \mci{1} Z_1 \mci{2} Z_2 ... \mci{m} Z_m \rb \lb W \lc Y \rb
                \Rightarrow^*
                w_i ... w_p w_{p + 1} ... w_j = w
            \]
    \end{enumerate}
\end{proof}

\begin{prop}
    The algorithm is \textbf{complete}: if $w \in L(G)$, then $(S, 1, k) \in P$.
\end{prop}
\begin{proof}
    We have to prove the left-to-right direction of the invariant:
    \begin{equation}\label{eq:right}
        \lb X \rb \Rightarrow^* w_i ... w_j \implies (X, i, j) \in P
    \end{equation}
    From which would follow $ \lb S \rb \Rightarrow^* w_1 ... w_k \implies (S, 1, k) \in P$,
    which is what we need to prove.

    Let $\mu: \lb X \rb \Rightarrow \alpha \Rightarrow^* w_i ... w_j$
    be a derivation for $G$. We will prove (\ref{eq:right}) by induction over
    $\mu$ from right to left.

    The only way for $\lb X \rb$ to appear to the left
    of a production is $\lb X \rb \rightarrow \alpha$ - we can now look at
    all ways for this to happen.
    \begin{enumerate}
        \item $\alpha = a \in \Sigma, X \in f(a)$. Then $w = \alpha, i = j, w_i = a$,
            which means $(X, i, i) = (X, i, j) \in P$.
        \item $X = W \mci{1} Z_1 \mci{2} Z_2 ... \mci{m} Z_m$ and
            $\alpha = \lb W \rc Y \rb \lb Y \mci{1} Z_1 \mci{2} Z_2 ... \mci{m} Z_m \rb$.
            According to Proposition \ref{prop:concat}, there exists $p$ such that
            \begin{itemize}
                \item $\lb W \rc Y \rb \Rightarrow^* w_i ... w_p$
                \item $\lb Y \mci{1} Z_1 \mci{2} Z_2 ... \mci{m} Z_m \rb \Rightarrow^* w_{p+1} ... w_j$
            \end{itemize}
            However, according to the inductive hypothesis, this means that
            $(W \rc Y, i, p) \in P$ and
            $(Y \mci{1} Z_1 \mci{2} Z_2 ... \mci{m} Z_m, p + 1, j) \in P$. Thus,
            by the algorithm definition,
            $(X = W \mci{1} Z_1 \mci{2} Z_2 ... \mci{m} Z_m, i, j) \in P$.
        \item $X = W \mci{1} Z_1 \mci{2} Z_2 ... \mci{m} Z_m$ and
            $\alpha = \lb Y \mci{1} Z_1 \mci{2} Z_2 ... \mci{m} Z_m \rb \lb W \lc Y \rb$
            (analogous to \ref{eq:right}).
            According to Proposition \ref{prop:concat}, there exists $p$ such that
            \begin{itemize}
                \item $\lb Y \mci{1} Z_1 \mci{2} Z_2 ... \mci{m} Z_m \rb \Rightarrow^* w_i ... w_p$
                \item $\lb W \rc Y \rb \Rightarrow^* w_{p + 1} ... w_j$
            \end{itemize}
            However, according to the inductive hypothesis, this means that
            $(W \lc Y, p + 1, j) \in P$ and
            $(Y \mci{1} Z_1 \mci{2} Z_2 ... \mci{m} Z_m, i, p) \in P$. Thus,
            by the algorithm definition,
            $(X = W \mci{1} Z_1 \mci{2} Z_2 ... \mci{m} Z_m, i, j) \in P$.
    \end{enumerate}

\end{proof}

\subsubsection{The Vijay-Shanker algorithm}

This algorithm behaves like the categorial CYK algorithm for derivations which use only
short categories. In order to pack longer categories in a way which would not
result in exponential complexity, it uses some clever tricks.

A detailed explanation and proof can be found in the original paper \cite{shanker}.

Let $ G = \langle \Sigma, N, S, f, n \rangle $ be a CCG and $w = w_1 ... w_k$
be a word.

Let $c' = max \{ |X| \mid X \in f(a), a \in \Sigma \}$ and $c = max \{ c', n \}$.

The algorithm uses two types of items:
\begin{enumerate}
    \item $(X, i, j), X \in C(N), 1 \leq i \leq j \leq k$
    \item $(A, \xi, X, T, i, j, p, q), A \in N, T \in \{ \lc, \rc \} N, \xi \in \{ \lc, \rc \}, X \in C(N), 1 \leq i \leq p \leq q \leq j \leq k$
\end{enumerate}

As with the categorial CYK algorithm, we build a set ${P}$ of items and aims to produce the item
$(S, 1, k)$.

\begin{enumerate}
    \item If $X \in f(w_i)$, then $(X, i, i) \in P$
    \item If $(X \rc B, i, t) \in P, (B \mci{1} Z_1 \mci{2} Z_2 ... \mci{m} Z_m, t + 1, j) \in P$,
        $|X \mci{1} Z_1 \mci{2} Z_2 ... \mci{m} Z_m| \leq c$
        then $(X \mci{1} Z_1 \mci{2} Z_2 ... \mci{m} Z_m, i, j) \in P$
    \item If $(A \mci{*} X \rc B, i, t) \in P, (B \mci{1} Z_1 \mci{2} Z_2 ... \mci{m} Z_m, t + 1, j) \in P$,
        $|A \mci{*} X \mci{1} Z_1 \mci{2} Z_2 ... \mci{m} Z_m| > c$
        then $(A, \mci{1}, Z_1 \mci{2} Z_2 ... \mci{m} Z_m, \rc B, i, j, i, t) \in P$
    \item If $(A, \xi, X \rc B, T, i, t, p, q) \in P$,
        $(B \mci{1} Z_1 \mci{2} Z_2 ... \mci{m} Z_m, t + 1, j) \in P$,
        $m > 1$, then $(A, \xi, Z_1 \mci{2} Z_2 ... \mci{m} Z_m, \rc B, i, j, i, t) \in P$.
    \item If $(A, \rc, B, T, i, t, p, q) \in P$,
        $(B \mci{1} Z_1 \mci{2} Z_2 ... \mci{m} Z_m, t + 1, j) \in P$,
        $m > 1$, then $(A, \mci{1}, Z_1 \mci{2} Z_2 ... \mci{m} Z_m, T, i, j, p, q) \in P$.
    \item If $(A, \xi, X \rc B, T, i, t, p, q) \in P$,
        $(B \mci{*} Z, t + 1, j) \in P$,
        then $(A, \xi, X \mci{*} Z, T, i, j, p, q) \in P$.
    \item If $(A, \xi, X \rc B, T, i, t, p, q) \in P$,
        $(B, t + 1, j) \in P$,
        then $(A, \xi, X, T, i, j, p, q) \in P$.
    \item If $(A, \rc, B, T, i, t, p, q) \in P$,
        $(B, t + 1, j) \in P$, $(A \mci{1} X_1 \mci{2} X_2 ... \mci{m} X_m T, i, j) \in P$,
        then $(A \mci{1} X_1 \mci{2} X_2 ... \mci{m} X_m, i, j) \in P$.
    \item If $(A, \rc, B, T, i, t, p, q) \in P$,
        $(B, t + 1, j) \in P$, $(A, \mci{1} X_1 \mci{2} X_2 ... \mci{m} X_m T, T', i, j, r, s) \in P$,
        then $(A, \mci{1}, X_1 \mci{2} X_2 ... \mci{m} X_m, T', i, j, r, s) \in P$.
\end{enumerate}

The other 8 rules regarding backward composition
($2'$ through $9'$) are analogous to their forward counterparts, but with
flipped indices.

The invariant here is composed of two parts:

\begin{itemize}
    \item $(X, i, j) \in P \iff |X| \leq c \land \lb X \rb \Rightarrow^* w_i ... w_j$
    \item $(A, \xi, X, \mci{*} B, i, j, p, q) \in P$
        exactly when there exists $\nu \in \{ \lc, \rc \}, Y \in C(N)$ such that
        \begin{itemize}
            \item $1 \leq |Y| \leq c'$
            \item $\lb A \nu Y \xi X \rb \Rightarrow^* w_i ... w_{p - 1} \lb A \nu Y \mci{*} B \rb w_{q + 1} ... w_j$
            \item $\lb A \nu Y \mci{*} B \rb \Rightarrow^* w_p ... w_q$
        \end{itemize}
\end{itemize}

For smaller grammars, which rarely generate very long categories during
parsing, this algorithm is in fact slower than simple CYK-based alternatives.
The strength of this algorithm comes when parsing automatically-generated
grammars from treebanks.

For this reason, this algorithm will not be used. For a more indepth
overview, see \cite{shanker}.


\end{document}
