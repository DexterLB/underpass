\documentclass[main.tex]{subfiles}
\begin{document}

\subsection{The Intermediate Language (Minipass)}
\label{sec:minipass}
Overpass is a relatively complex language with assignment, sequentiality
and specialised syntax. As such, it is not well-suited for performing
AST transofmations.

Minipass addresses this problem: its goal is to be easy to manipulate
and reason about, and be translatable to Overpass. It is based on the typed
lambda calculus variant described in \cref{sec:lambda} and speaks about a
world much simpler than the one described by Overpass - a directed graph
with labeled edges and vertices.

The graph abstraction works as follows:
\begin{itemize}
    \item Overpass objects (nodes, ways, relations, areas) are regarded as
        vertices
    \item Object tags are encoded into vertex labels - for example, we
        may have a label for "named London" or "is a cafe".
    \item Edges labels represent relations between objects: for example,
        we may have an edge that means "nearby" or an edge "on the same
        transport route"
\end{itemize}

\begin{example}
    To model a query like "\code{bus stops near schools in Russia}"
    using the graph abstraction, we could think like this:
    \begin{itemize}
        \item Let $B$ be the set of all vertices whose label unifies with
            "\code{is a bus stop}"
        \item Let $S$ be the set of all vertices whose label unifies with
            "\code{is a school}"
        \item Let $R$ be the set of all vertices whose label unifies with
            "\code{is called Russia}"
        \item Let $N$ be the set of all vertices we can reach from $S$
            by traversing an edge whose label unifies with "\code{near to}"
        \item Let $P$ be the set of all vertices we can reach from $R$
            by traversing an edge whose label unifies with "\code{is inside of}"
        \item Our result is $B \cup N \cup P$
    \end{itemize}
\end{example}

\subsubsection{Language grammar}
\label{minipass:grammar}
Here is a BNF grammar for the intermediate language.
\begin{grammar}
    <program> ::= <tstatement> <tstatement>*

    <tstatement> ::= <statement> `.'

    <statement>  ::= <import-statement>
           \alt <term-definition>
           \alt <type-definition>

    <import-statement> ::= `import' <filepath>
    <term-definition>  ::= <identifier> `:=' <term>

    <type-definition>  ::= <identifier> `<' <type>

    <type> ::= <simple-type> | <type> <arrow> <simple-type>

    <simple-type> ::= <identifier> | `*' | `(' <type> `)'

    <arrow> ::= `->'

    <term> ::= <nonapp-term> | <nonapp-term> <term>

    <nonapp-term> ::= <constant> | <variable> | <constr> | <abstr> | `(' <term> `)'

    <constant> ::= <identifier> | <literal>

    <variable> ::= <identifier>

    <constr> ::= <type> `[' <term> `]'

    <abstr> ::= <lambda> <declaration> `=>' <term>

    <lambda> ::= `\\' | `lambda'

    <declaration> ::= <variable> | <variable> `:' <type>

    <literal> ::= <string-literal> | <num-literal>

    <string-literal> ::= <single-quoted-string>

    <number-literal> ::= <decimal-number>

    <filepath> ::= <double-quoted-string>
\end{grammar}

The definitions for \syntax{<single-quoted-string>},
\syntax{<double-quoted-string>}
and \syntax{<number-literal>}
have been omitted: the first two are plain quoted strings with backslash escaping,
while the third is a standard floating point number.

Note that Minipass is characterised by its predefined constants and basic
types. Because of the modular architecture of the implementation, they can be
replaced by something else, allowing to target a different intermediate language.

Applications are left-associative while types are right-associative. Comments
can be entered by using the \code{--} syntax (everything on the same line after
the two dashes is ignored by the parser.

Import statements are used to split a program into multiple files, and are
resolved relative to the current program's path (or the current working
directory when not applicable).

\subsubsection{Builtins and base types}\label{sec:minipassbasetypes}
Since Minipass is built on top of $\lambda$-calculus syntax, all built-in
"functions" are actually \emph{constants} in the context of $\lambda$-terms.

The base types are:
\begin{center}
    \begin{tabular}{r p{0.7\textwidth}}
        \code{Num} & represents a floating point number. \cendrow
        \code{String} & represents a string of unicode characters. \cendrow
        \code{List} & represents a linked list of \code{Num}s, \code{String}s and
            nested \code{List}s. \cendrow
        \code{GSet} & refers to a set of geographical objects. It is not represented
            internally in any way: the only way to touch \code{GSet} objects is
            via builtin operations that work on \code{GSet}s. \cendrow
    \end{tabular}
\end{center}

The implementation also allows for the special type \code{*} that
acts as a wildcard. Typeless declarations are assumed to bear the wildcard type,
which is removed from all terms by performing type inference (see \cref{sec:typeinf})
before translation.

The builtin identifiers (which act as constants in $\lambda$-terms) are the
following:
\begin{center}
    \begin{tabular}{r l p{0.5\textwidth}}
        Identifier  & Type & \\
        \hline
        \code{and}  & $GSet \tot GSet \tot GSet$ & set intersection \cendrow
        \code{or}   & $GSet \tot GSet \tot GSet$ & set union \cendrow
        \code{not}  & $GSet \tot GSet$ & set negation \cendrow
        \hline
        \code{get}  & $List \tot GSet$ & get geographical elements by vertex
            label \cendrow
        \code{next} & $List \tot GSet \tot GSet$ & traverse edges by label,
            obtaining a new \code{GSet} \cendrow
        \hline
        \code{empty} & $List$ & empty list \cendrow
        \code{consNum} & $Num \tot List \tot List$ & prepend a \code{Num} to
            a list \cendrow
        \code{consString} & $String \tot List \tot List$ & prepend a \code{String} to
            a list \cendrow
        \code{consList} & $List \tot List \tot List$ & prepend a \code{List} to
            a list (prepends the list itself, not its elements)\cendrow
    \end{tabular}
\end{center}

Should the need arise, the addition of extra operations (for example arithmetic)
is trivial.

\subsubsection{Composing terms}
The syntax for composing terms is analogous to $\lambda$-calculus syntax
(\cref{sec:lambda}).
The valid term constructions are the following:
\begin{itemize}
    \item Constant: number literal, string literal or builtin identifier
    \item Variable: any identifier that is not a builtin
    \item Application: term application is represented by listing terms
        separated by whitespace. It is by default left-associative and
        application order may be influenced by using round braces.
    \item Abstraction:
        \begin{center}
            \syntax{\code{\textbackslash}<variable>\code{:}<type>\code{ => }<term>.} \\
            or\\
            \syntax{\code{\textbackslash}<variable>\code{ => }<term>.}
        \end{center}
    \item Construction:
        \begin{center}
            \syntax{<type>\code{[}<term>\code{]}}
        \end{center}
\end{itemize}

\subsubsection{Labels and graph queries}

The main use for \code{List}s is as \emph{labels}: all vertices and edges can be thought
to be labelled with a (possibly infinite) set of such labels. Another way
to think about labels is as \emph{queries} that retrieve data (either new
data in the case of vertice labels, or relational data in the case of edge
labels).

Although for the sake of simplicity the implemented parser doesn't support
a list literal shorthand, for the sake of examples we will adopt the syntax
such that
\begin{lstwrap}\begin{lstlisting}
    [[42, 'bar'], 'foo']
\end{lstlisting}\end{lstwrap}
means
\begin{lstwrap}\begin{lstlisting}
    consList
        (consNum 42 (consString 'bar' empty))
        (consString 'foo' empty)
\end{lstlisting}\end{lstwrap}

Here are the two edge label templates that are currently implemented in the
translator:

\begin{center}
    \begin{tabular}{ p{0.45\textwidth} p{0.45\textwidth} }
        Label & Semantics \\
        \hline
        \code{['around', [<dist>]]} & Items that are within \code{<dist>} metres
        from given items \cendrow
        \code{['in']} & Items that are \emph{inside} the given items \cendrow
    \end{tabular}
\end{center}

And the three vertex label templates:

\begin{center}
    \begin{tabular}{ p{0.45\textwidth} p{0.45\textwidth} }
        Label & Semantics \\
        \hline
        \pbox{0.4\textwidth}{\vspace{1em} \code{['tagFilter',} \\ \code{  ['=', <key>, <value>]]}} & Items that contain a
            tag that has the given key and value \cendrow
        \pbox{0.4\textwidth}{\vspace{1em} \code{['tagFilter',} \\ \code{  ['~', <key>, <regex>]]}} & Items that contain a
            tag that has the given key and whose value matches the given
            regular expression \cendrow
        \code{['all', [<type>]]} & All items in the universe that are of
            the given Overpass type (way, relation, area, node) \cendrow
    \end{tabular}
\end{center}

\begin{example}
\label{example:schools}
    Here is a query that retrieves all schools in Bristol:
    \begin{lstwrap}\begin{lstlisting}
and
    (get (consString 'tagFilter'
            (consList (consString '='
                (consString 'amenity'
                    (consString 'school' empty)))
        empty)))
    (next (consString 'in' empty)
        (get (consString 'tagFilter' (consList
            (consString '='
                (consString 'name'
                    (consString 'Bristol' empty)))
        empty))))
    \end{lstlisting}\end{lstwrap}
\end{example}

Since Minipass queries can get too long for humans to read/write very quickly,
a mechanism for modularisation has been implemented: Just like in most
programming languages, the expert can define
named terms and types that can later be composed into other terms and types.

\subsubsection{Defining types}\label{sec:definingtypes}
New types are defined by subtyping an existing type via a "type definition"
(subtype assertion) that has the following syntax:
\begin{center}
    \code{NewType < OldType.}
\end{center}
where \code{NewType} is not a supertype of \code{OldType} and has not been
already defined.

The subtyping semantics
are the same as the ones described in \cref{sec:lambda}: all subtype assertions
provided by the expert
make up a subtype library, which is checked for validity and used for resolving
the types contained in all terms.

\subsubsection{Defining terms}
New terms are defined by using the "term definition" syntax:
\begin{center}
    \code{term_name := some term.}
\end{center}

Named terms are resolved after parsing by performing free variable substitution
in the source code and are bundled in a "term library". The resolver also
handles possible type errors and loops.

Note that this language is not turing complete (not even primitively recursive)
since recursion is not handled\footnote{
    Actually, recursion is easy to implement (and was implemented at some point)
    by adding a recursion combinator
    and resolving term definition loops to constructions that use that combinator.
    However, so far I haven't found any use for recursion in such a language
    apart from abuse.
}.

Note that \emph{only closed terms} are permitted in definition statements
(free variables make little sense).

\subsubsection{A "standard" library}
\label{stdlib}

This is a small library of term definitions that makes some basic queries
easier to write:
\begin{lstwrap}\begin{lstlisting}
-- identity
id        := \y => y.

-- the set of all cities
city      := or (or
        (kv 'admin_level' '8')
        (kv 'admin_level' '6'))
    (kv 'capital' '4').

-- within 100 metres
near      :=  within 100.

-- find items by name
name      :=  \x => or (or
        (kv 'name' x)
        (kv 'int_name' x))
    (kv 'name:en' x).

-- find items by name infix
nameLike  :=  \x => or (or
        (contains 'name' x)
        (contains 'int_name' x))
    (contains 'name:en' x).

-- find items by amenity key (fountain, school...)
amenity   :=  kv 'amenity'.

-- find items by tag key and value
kv        :=  \k: String, n: String => get
    (consString 'tagFilter'
        (consList (consString '='
            (consString k (consString n empty)))
    empty)).

-- find items by tag key and value infix
contains  :=  \k: String, n: String => get
    (consString 'tagFilter'
        (consList (consString '~'
            (consString k (consString n empty)))
        empty)).

-- find items within distance of given items
within    :=  \dist : Num => next
    (consString 'around' (consNum dist empty)) .

-- find items inside of given items
in        :=  next (consString 'in' empty).

-- get all items of given type
nodes     :=  get (consString 'all'
    (consString 'nodes' empty)).
ways      :=  get (consString 'all'
    (consString 'ways' empty)).
relations :=  get (consString 'all'
    (consString 'relations' empty)).
areas     :=  get (consString 'all'
    (consString 'areas' empty)).

-- get all items in the universe
everything := get (consString 'all' empty).
\end{lstlisting}\end{lstwrap}

\begin{example}
    With the aid of this term library, the query in \cref{example:schools}
    becomes more concise:

    \begin{lstwrap}\begin{lstlisting}
and (amenity 'school') (in (name 'Bristol'))
    \end{lstlisting}\end{lstwrap}
\end{example}

When regarding a standalone minipass program, the term \code{main} is used
as a query.

\end{document}

