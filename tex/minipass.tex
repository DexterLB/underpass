\documentclass[main.tex]{subfiles}
\begin{document}

\subsection{The Intermediate Language}
Note: this is very much subject to change.

Let our world be a directed graph whose vertices we call \emph{elements}.
Each \emph{edge} carries a \emph{label}. Each \emph{label} is a heterogenous
\emph{List} of \emph{Numbers}, \emph{Strings} and other \emph{Lists}.

\begin{grammar}

<term> ::= <variable>
\alt <constant>
\alt <term> <term>
\alt `lambda' <declaration> `{' <term> `}'
\alt `(' <term> `)'

<constant> ::= <stringliteral>
          \alt <numberliteral>
          \alt `and'
          \alt `or'
          \alt `get'
          \alt `next'
          \alt `empty'
          \alt `consNum'
          \alt `consString'
          \alt `consList'


<variable> ::= <identifier>

<declaration> ::= <variable> `:' <type>
\alt <variable>

<type> :: <type> `->' <type>
\alt <basictype>
\alt `(' <type> `)'

<basictype> ::= `Set'
\alt `Num'
\alt `String'
\alt `List'
\alt `*'

<identifier> ::= [a-zA-Z_][a-zA-Z_0-9]*

<numberliteral>     ::= -?[0-9]+(\textbackslash.[0-9]+)?

<stringliteral>     ::= <todo>
\end{grammar}

For ease of writing, let us consider that types are right-associative,
applications are left-associative and that we can write list literals
within square brackets:
\begin{lstlisting}
[42, "foo"] := consNum 42 (consString "foo" empty)
\end{lstlisting}

And here are the types for all constants:

\begin{lstlisting}
<stringliteral> : String
<numberliteral> : Num

and       : Set -> Set -> Set   -- set intersection
or        : Set -> Set -> Set   -- set union
not       : Set -> Set          -- set negation

get       : List -> Set         -- get elements by label
next      : List -> Set -> Set  -- traverse edges

-- constructing lists
empty       :                     List
consNum     : Num      -> List -> List
consString  : String   -> List -> List
consList    : List     -> List -> List

\end{lstlisting}

With this abstraction, we can represent most of the Overpass queries for
constructing sets (filters, recurse, in/is_in, set operations, conditions)
by encoding them as labels. Here are some examples:

\begin{lstlisting}
within : Num -> Set -> Set
    := lambda dist { next ["around", dist] }

onStreet : String -> Set
    := lambda str  { next [
        "tagFilter", ["==", "addr:street", str]] }
\end{lstlisting}

\end{document}

