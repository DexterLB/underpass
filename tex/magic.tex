\documentclass[main.tex]{subfiles}
\begin{document}

\subsection{Of mythical creatures and magical sets}

Sooner or later, in every work related to formal logic ariese the need for
a set of basic objects. For example, we want construct a system for basic
arithmetic, for which we need "variables".
At a first glance, this is a trivial matter - just take an arbitrary countable
set and call it a "set of variables" - but wait! What if the left parenthesis
symbol ends up in this set? Well, let's fix that: we banish all parentheses
from our set of variables $\mathbb{V}$.
\[ \{ (, ) \} \cap \mathbb{V} = \varnothing \]
Fine. But what if the number $7$ ends up in our set of variables? Then,
when we say 7, did we mean the number or the variable? Well. No place for
numbers in our set of variables as well.
\[ ( \{ (, ) \} \cup \mathbb{N} ) \cap \mathbb{V} = \varnothing \]

But wait! What if the string $(42)$ ends up in our set of variables? That will
surely lead to problems. Try again:
\[ ( \{ (, ) \} \cup \mathbb{N} )^* \cap \mathbb{V} = \varnothing \]

Alas, this is still not enough. We haven't excluded strings like $foo(bar$.

Hmm, maybe this problem isn't that easy after all - why not give up and make
$\mathbb{V}$ be the set of all splotches of ink which an ancient greek would
recognise as a sequence of their letters?

As entertaining as it is to think about this problem instead of trying to write
the actual thesis, it's time for a quick solution. One way to wave hands
about this is to assume that $\mathbb{V}$ is a magical set which contains
no elements that we have used anywhere: essentially a set which is disjoint
with any set we feel convenient that it be disjoint with. It contains none of
the special symbols we use elsewhere.

Another way is to construct a very specific set, which is what we'll do here:

\[ \mathbb{V} = \{ \mname{v}, \mname{v'}, \mname{v''} ... \} \]

\end{document}
