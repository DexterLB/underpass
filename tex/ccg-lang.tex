\documentclass[main.tex]{subfiles}
\begin{document}
\pagebreak
\subsection{The CCG definition language}
When paired with Minipass, the language for defining CCGs consists of
the entire Minipass language
(\cref{minipass:grammar}) extended by additional syntax for defining
token matchers and categories. It can also be used standalone as a library
(in that case, arbitrary payloads can be provided instead of Minipass
terms and types).

\begin{grammar}
    <program> ::= <tstatement> <tstatement>*

    <tstatement> ::= <statement> `.'

    <statement>  ::= <import-statement>
           \alt <begin-statement>
           \alt <match-statement>
           \alt <term-definition>
           \alt <type-definition>

    <import-statement> ::= `import' <filepath>

    <begin-statement>  ::= `begin' <category>

    <match-statement>  ::= <phrase-rule> `:' <target> (`,' <target>)*

    <phrase-rule> ::= <match-rule> (`<' <phrase-id> `>' <match-rule>)*

    <match-rule> ::= <matcher> (<template> | $\varepsilon$)

    <matcher> ::= <simple-matcher> | <or-matcher> | <and-matcher>

    <not-or-matcher> ::= <simple-matcher> | <and-matcher>

    <not-and-matcher> ::= <simple-matcher> | <or-matcher>

    <simple-matcher> ::= <exact-matcher> | <regex-matcher> | `(' <matcher> `)'

    <or-matcher> ::= <not-or-matcher> `|' <matcher>

    <and-matcher> ::= <not-and-matcher> `\&' <matcher>
    
    <exact-matcher> ::= <identifier> `=' <double-quoted-string>
    
    <regex-matcher> ::= <identifier> `~' <double-quoted-string>

    <template> ::= <backtick-quoted-string> `:' <type>

    <target> ::= <category> `@' <term>

    <category> ::= <simple-category> | <simple-category> <slash> <category>

    <simple-category> ::= <type> | `(' <category> `)'

    <slash> ::= `/' <modality> | `\\' <modality>

    <modality> ::= `.' | `d' | `x' | `*' | $\varepsilon$
\end{grammar}


Constructions for defining terms, subtyping, importing files, comments and such
remain. What the CCG definition language adds is the ability to match natural
language tokens and assign categories and terms to them, and the ability to
select a starting category.

The starting category is is specified with the following statement:
\begin{center}
    \code{begin SomeCategory.}
\end{center}
There can be only one \code{begin} statement in the entire program.

\subsubsection{Matching tokens}
Tokens (words) in the natural language input are emitted by the
tokeniser as sets of
key-value data (for example, the token \code{cities} may be represented as
\code{raw=cities,pos=nnp,lemma=city} - for a more detailed explanation
see \cref{sec:tokenisation}). The CCG definition language allows us to define
\emph{matchers} which can either match a token or not match a token.

The following matchers have been implemented:

\begin{itemize}
    \item exact matcher (matches tokens which contain a tag with key `key'
        and value `value'):
        \begin{center}
            \code{key='value'}
        \end{center}
    \item regex matcher (matches tokens which contain a tag with key `key'
        and a value which matches the regular expression `regex'):
        \begin{center}
            \code{key~'regex'}
        \end{center}
    \item "and" matcher (matches tokens that are matched by both given matchers):
        \begin{center}
            \syntax{<matcher>}\code{ \& }\syntax{<matcher>}
        \end{center}
    \item "or" matcher (matches tokens that are matched by any of the given matchers):
        \begin{center}
            \syntax{<matcher>}\code{ | }\syntax{<matcher>}
        \end{center}
\end{itemize}

\subsubsection{Assigning categories and terms to matched tokens}
The syntax for assigning categories and terms is analogous to the one described
in \cref{shorthand}:
\begin{center}
    \syntax{<matcher>}\code{:}\syntax{<category>}\code{@}\syntax{<payload>}
\end{center}
where \syntax{<payload>} is a Minipass term when the language is used to generate
Minipass.

Multiple categories and respective terms may be attached to a single
matcher by separating them with commas:
\begin{center}
    \syntax{<matcher>}\code{:}\syntax{<category>}\code{@}\syntax{<payload>},
    \syntax{<category>}\code{@}\syntax{<payload>}, ...,
    \syntax{<category>}\code{@}\syntax{<payload>}
\end{center}

The payload terms can only be closed terms and their type must match the
respective category's type.

Categories are constructed as follows:
\begin{itemize}
    \item Every Minipass type is a valid atomic category (including types defined
        by the expert via subtyping). Using a complex type for an atomic
        category is permitted but is seldom useful.
    \item Complex categories may be constructed by the syntax
        \begin{center}
            \syntax{<category> <slash> <category>}
        \end{center}
        where \syntax{<slash>} is a modal slash (\code{/.}, \code{/s},
        \code{/d} or \code{/x} which respectively stand for $\rci{\moddot}$,
        $\rci{\modstar}$, $\rci{\modr}$, $\rci{\modx}$, and their backslash
        counterparts). The modality may be omitted, and is then assumed to be
        the dot modality.
    \item Categorial variables (\cref{hack:catvars}) are specified with a
        dollar sign followed by \code{[a-zA-Z][a-zA-Z0-9\_]*}
\end{itemize}

\begin{exampleenv}
Here is a grammar defined using the CCG definition language\footnote{
    Here \code{"library.ccg"} is assumed to contain the code from \cref{stdlib}
}:
\begin{lstlisting}
import "library.ccg"

-- Use Set as the starting category.
-- This means that all queries parsed
-- by this CCG will have type Set.
begin Set.

-- Here Mult is used to handle lists like
-- "schools, churches and hospitals"
Mult < Set.

raw="and"
    : Set / Set \ Set @ or
    , Mult / Set \ Set @ \r, l => Mult[or l r].

pos="comma" : $X \ $X @ id -- some commas mean nothing
            , Set \ Set / Mult @ or.

raw="in"
    : Set / Set \ Set
    @ \things, where => and things (in where).

lemma="school",pos="nnp" : Set @ amenity 'school'.
\end{lstlisting}
\end{exampleenv}

\subsubsection{Composing phrases}
In most cases we would would want our rule to match single tokens. Sometimes,
however, it is convenient to be able to match an entire phrase (for example
phrases like `parking lot', `coworking space' or `rubbish bin') to assign
categories to.

To do this, the CCG definition language supports a syntax for composing phrases
(one way to think of it is like ``matcher concatenation''):
\begin{center}
    \syntax{<matcher>} \code{<>} \syntax{<matcher>} \\
    or \\
    \syntax{<matcher>} \code{<DummyCategory>} \syntax{<matcher>} \\
\end{center}
This phrase syntax is a shorthand for creating a dummy category
for each pair and assigning categories such that the entire phrase will
be assigned the correct category and term specified for it.

The expert can also explicitly specify the name of the dummy category
(this is mainly useful for debugging purposes: the category will show up
in the resulting derivation trees).

For more details about this construction, see \cref{hack:phrases}.

\begin{exampleenv}
The following statement:
\begin{lstlisting}
lemma="place" <> raw="of" <> lemma="worship"
    : Set
    @ amenity 'place_of_worship'.
\end{lstlisting}

Desugars to something like this:
\begin{lstlisting}
Dummy1 < * -> *.
Dummy2 < * -> *.

lemma="place"
    : Set /s Dummy2 /s Dummy1
    @ \x: Dummy1, y: Dummy2 =>
        amenity 'place_of_worship'.

raw="of"
    : Dummy1
    @ Dummy1[\x => x].

lemma="worship"
    : Dummy2
    @ Dummy2[\x => x].
\end{lstlisting}
\end{exampleenv}

\subsubsection{Term templates}
Sometimes it is required to insert arbitrary information from tokens into
the respective generated terms (e.g. decimal numbers, names, etc.).

This is currently implemented by adding a \emph{template} after the
matcher. Templates have the following syntax:
\begin{center}
    \syntax{\code{\textasciigrave}<template-content>\code{\textasciigrave : }<type>}
\end{center}

They are stored in raw form instead of being
parsed at the time of parsing the
CCG definition source. Apart from regular term syntax, templates may also
contain \emph{template variables}, the syntax for which is a dollar sign
followed by a valid key that may be found in token data.

Whenever a rule with a template matches a token, template variables are
substituted by their values, the template is parsed as a term and is fed
as a first argument to the primary term specified in the rule.

Templates are inherently unsafe - failure to parse the template after substitution
or having a term with a wrong type produced by the template will cause the
CCG parsing process to fail. It is the expert's responsibility to ensure that
templates are correct.

\begin{exampleenv}
Consider the following rule which contains a template:
\begin{lstlisting}
pos="nnp" `'$raw'` : String
    : Set
    @ name.
\end{lstlisting}
It is valid at compile time because the term \code{name} has type
$String \tot Set$, which yields a term of type $Set$ after being given
a $String$ as argument.
When matching the token data "\code{pos="nnp",raw="Belgium"}", the template
\code{ \textasciigrave'\$raw'\textasciigrave } will become
\code{ \textasciigrave'Belgium'\textasciigrave } after substitution. Its type
(\code{String}) will be verified, and the resulting term
(\code{name 'Belgium'}) will be constructed.
\end{exampleenv}

In the case of multiple concatenated matchers with templates, all templates
are fed to the term from left to right. This is achieved by making the
generated dummy categories subtypes of the respective template types instead
of the identity type. The resulting term then forms naturally.
\end{document}
