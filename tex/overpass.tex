\documentclass[main.tex]{subfiles}
\begin{document}
\subsection{Overview of the Overpass language}
The Overpass language \cite{overpass} has been created by the OpenStreetMap
community in order to be able to make more complex queries to map data. 

Overpass queries can be written in two ways: XML-based queries and OverpassQL-based
queries. The XML OverpassQL frontend has been better-maintained, has more features
and is easier to generate programmatically - thus, it has been chosen as a target
language for this project.

This section will present a subset of Overpass QL that is relevant to this project: for a
more complete overview, refer to \cite{overpassql}.

\subsubsection{Basic language structure}
OverpassQL consists of a list of \emph{statements} which retreive, manipulate
or output OpenStreetMap data.

The map data consists of \emph{objects} where each object has a set of key-value
tags (keys and values are strings but some values may encode a more complex
structure within a string).

An object viewed from Overpass may be one of the following types:
\begin{itemize}
    \item \emph{node} - a point on the map, identified by latitude, longitude
        and a node ID. Nodes represent point-like features like trees, fountains,
        businesses, mountain peaks, etc.
    \item \emph{relation} - a grouping of multiple objects by some common
        feature. Examples of relations are:
        \begin{itemize}
            \item all stops in a bus route
            \item all intersections of a highway
            \item poles of a power grid line
        \end{itemize}
        Relations may contain nodes, ways and other relations. Each member of
        a relation may have an optional \emph{role}, which is a text field
        that may be used to further group relation elements.
    \item \emph{way} - an ordered list of Nodes which is used for defining
        continous features. Ways may be open or closed (closed ways have an
        identical first and last node). Open ways are used for defining things
        like roads, rivers and railways, whilc closed ways are used for defining
        polygons (building boundaries, area borders, closed features).
    \item \emph{area} - not an actual OpenStreetMap object. Closed ways or relations
        which contain closed ways may also be classified as an area by some
        criteria (one of which is the presence of an \code{area=yes} tag).
        Areas allow additional operations to the ones usable only for ways
        taking advantage of a cached optimised representation of each polygon.
\end{itemize}

The abstraction Overpass uses to represent map data is called a \emph{set}, 
where a set can contain any number of objects. Sets are heterogenous and thus
there is no limit on the variety of objects that may reside in a single set.

\fixme{finish this}
\end{document}
