\documentclass[main.tex]{subfiles}
\begin{document}
\subsection{Overview of the Overpass language}
The Overpass language \cite{overpass} has been created by the OpenStreetMap
community in order to be able to make more complex queries to map data.

Overpass queries can be written in two ways: XML-based queries and OverpassQL-based
queries. The XML OverpassQL frontend has been better-maintained, has more features
and is easier to generate programmatically - thus, it has been chosen as a target
language for this project.

This section will present a subset of Overpass QL that is relevant to this project: for a
more complete overview, refer to \cite{overpassql}.

\subsubsection{Map data}
The universe which Overpass lets us examine is OpenStreetMap map data.
It consists of \emph{objects}, where each object has a set of key-value
tags (keys and values are strings but some values may encode a more complex
structure within a string).

An object may be one of the following types:
\begin{itemize}
    \item \emph{node} - a point on the map, identified by latitude, longitude
        and a node ID. Nodes represent point-like features like trees, fountains,
        businesses, mountain peaks, etc.
    \item \emph{relation} - a grouping of multiple objects by some common
        feature. Examples of relations are:
        \begin{itemize}
            \item all stops in a bus route
            \item all intersections of a highway
            \item poles of a power grid line
        \end{itemize}
        Relations may contain nodes, ways and other relations. Each member of
        a relation may have an optional \emph{role}, which is a text field
        that may be used to further group relation elements.
    \item \emph{way} - an ordered list of Nodes which is used for defining
        continuous features. Ways may be open or closed (closed ways have an
        identical first and last node). Open ways are used for defining things
        like roads, rivers and railways, while closed ways are used for defining
        polygons (building boundaries, area borders, closed features).
    \item \emph{area} - not an actual OpenStreetMap object. Closed ways or relations
        which contain closed ways may also be classified as an area by some
        criteria (one of which is the presence of an \code{area=yes} tag).
        Areas allow additional operations to the ones usable only for ways
        taking advantage of a cached optimised representation of each polygon.
\end{itemize}

\subsubsection{Basic language structure}
The abstraction Overpass uses to represent map data is called a \emph{set},
where a set can contain any number of objects. Sets are heterogeneous and thus
there is no limit on the variety of objects that may reside in a single set.

OverpassQL consists of a list of \emph{statements} which retrieve, manipulate
or output such sets. Sets may be stored in variables: the syntax for referring
to a variable is a dot followed by alphanumeric characters.
Statements are terminated by semicolons.
Each statement has an \emph{output set} and may have input sets. When unspecified,
an input or output set is implicitly set to the \emph{default set} (\code{.\_}).

There is one special statement - the ``\code{out}'' statement, which outputs
all objects from an input set to the user in some form. The output format may be set
via specific meta-statements or by API options - this, however, is not within
scope. In the Overpass Turbo web interface, results may be visualised as
dots and polygons on a map. The syntax of the ``\code{out}'' statement is:
\begin{lstwrap}\begin{lstlisting}
    out <input set>;
\end{lstlisting}\end{lstwrap}

\begin{example}
    The following statement saves all nodes with name Foo within the set variable
    \code{.\_}:
    \begin{lstwrap}\begin{lstlisting}
        node[name="Foo"];
        out;
    \end{lstlisting}\end{lstwrap}

    It is equivalent to the following statement:
    \begin{lstwrap}\begin{lstlisting}
        node[name="Foo"] -> ._;
        out ._;
    \end{lstlisting}\end{lstwrap}

    In order to use \code{foo} as an output set, we could do:
    \begin{lstwrap}\begin{lstlisting}
        node[name="Foo"] -> .foo;
        out .foo;
    \end{lstlisting}\end{lstwrap}
\end{example}

The most basic group of statements are \emph{retrieval statements}.
They may retrieve specific type of objects depending on what
\emph{object keyword} was used. Object keywords are the following:
\begin{center}
    \begin{tabular}{r|l}
        \code{node} & retrieves nodes \\
        \code{way} & retrieves ways \\
        \code{relation} & retrieves relations \\
        \code{area} & retrieves areas \\
        \code{nwr} & retrieves nodes, ways and relations \\
    \end{tabular}
\end{center}

To form a retrieval statement, an object keyword is followed by one or more
\emph{filters}:
\begin{lstwrap}\begin{lstlisting}
    <object keyword><filter><filter>...<filter>
\end{lstlisting}\end{lstwrap}
Objects which match \emph{all} filters are stored in the output set.

\subsubsection{Tag filters}
The most basic group of statements retrieve objects depending on whether a
specific filter matches its tags. All tag filters are specified within
square braces.

These are the possible tag filters\footnote{All double quotes within filters
    may be replaced by single quotes (in order to escape double quotes within
    the value) or omitted (when the value contains only letters)}:
\begin{center}
    \begin{tabular}{|c|c|p{0.3\textwidth}|}
        \hline
        Filter & Negation & Effect \\
        \hhline{|=|=|=|}
        \code{["tag"]} & \code{[!"tag"]} & Matches if the object has a tag with key
            \code{tag} \\
        \hline
        \code{["tag"="content"]} & \code{[tag!="content"]}
            & Matches if the object has a tag with key \code{tag}
                and value \code{content} \\
        \hline
        \code{["tag"\textasciitilde"regex"]} & \code{[tag!\textasciitilde"regex"]}
            & Matches if the object has a tag with key \code{tag}
                and value that matches \code{regex} \\
        \hline
        \code{["tag"\textasciitilde"regex",i]} & \code{[tag!\textasciitilde"regex",i]}
            & Matches if the object has a tag with key \code{tag}
                and value that matches \code{regex}, ignoring case \\
        \hline
        \code{[\textasciitilde"regex1"\textasciitilde"regex2"]} & \code{[\textasciitilde"regex1"!\textasciitilde"regex2"]}
            & Matches if the object has a tag with key that matches \code{regex1}
                and value that matches \code{regex2} \\
        \hline
    \end{tabular}
\end{center}

\begin{example}
    Get nodes with amenity ``cafe'':
    \begin{lstwrap}\begin{lstlisting}
        node[amenity="cafe"];
        out;
    \end{lstlisting}\end{lstwrap}
    
    Get nodes with amenity ``cafe'' and name ``Stardeers'':
    \begin{lstwrap}\begin{lstlisting}
        node[amenity="cafe"][name="Stardeers"];
        out;
    \end{lstlisting}\end{lstwrap}

    Get nodes with amenity ``cafe'' and name ``Stardeers'', case insensitive:
    \begin{lstwrap}\begin{lstlisting}
        node[amenity="cafe"][name~"^Stardeers$",i];
        out;
    \end{lstlisting}\end{lstwrap}

    Get areas with an ``administrative'' tag:
    \begin{lstwrap}\begin{lstlisting}
        area["administrative"];
        out;
    \end{lstlisting}\end{lstwrap}
\end{example}

\subsubsection{Relative filters}
Some filters can use an input set to influence the retrieval query.
\begin{itemize}
    \item Set intersection: this filter only matches objects that are within
        the input set. The syntax is a dot followed by the input set's name.
        \begin{example}
            Get nodes with amenity ``cafe'':
            \begin{lstwrap}\begin{lstlisting}
                node[amenity="cafe"] -> .cafes;
                node.cafes;
            \end{lstlisting}\end{lstwrap}

            Get nodes with amenity ``cafe'' and name ``Stardeers'':
            \begin{lstwrap}\begin{lstlisting}
                node[amenity="cafe"] -> .cafes;
                node.cafes[name="Stardeers"];
            \end{lstlisting}\end{lstwrap}
        \end{example}
    \item Objects within an area: get objects that are physically inside
        of areas in the input set. The syntax uses the \code{area} keyword
        within round braces:
        \begin{lstwrap}\begin{lstlisting}
            (area.<input set>)
        \end{lstlisting}\end{lstwrap}
        \begin{example}
            Nodes in Frankfurt:
            \begin{lstwrap}\begin{lstlisting}
                area[name="Frankfurt"] -> fr;
                node(area.fr);
                out;
            \end{lstlisting}\end{lstwrap}
        \end{example}
    \item Physical distance: get objects within said distance from objects in
        the input set. The syntax involves using the \code{around} keyword
        within round braces:
        \begin{lstwrap}\begin{lstlisting}
            (around.<input set>:<radius in metres>)
        \end{lstlisting}\end{lstwrap}
        \begin{example}
            Cafes within 120m of a parking space:
            \begin{lstwrap}\begin{lstlisting}
                node[amenity="parking_space"] -> pspaces;
                node[amenity="cafe"](around.pspaces:120) -> x;
                out x;
            \end{lstlisting}\end{lstwrap}

            Cinemas within 100m of bus stops in Bonn:
            \begin{lstwrap}\begin{lstlisting}
                area[name="Bonn"];
                node(area)[highway=bus_stop];
                node(around:100)[amenity=cinema];
                out;
            \end{lstlisting}\end{lstwrap}
        \end{example}
\end{itemize}

\subsubsection{Set union}
Several statements that produce an output set may be chained within round
brackets. This is an \emph{union} statement and produces the union of all
output sets. Bare input sets (preceded by a dot) may also take part in a union.

\begin{example}
    Nodes and areas with an ``administrative'' tag:
    \begin{lstwrap}\begin{lstlisting}
        (node["administrative"]; area["administrative"];);
        out;
    \end{lstlisting}\end{lstwrap}

    Cafes in Frankfurt and Bonn:
    \begin{lstwrap}\begin{lstlisting}
        node[amenity="cafe"] -> .cafes;
        area[name="Bonn"] -> .bonn;
        area[name="Frankfurt"] -> .frankfurt;
        node.cafes(area.bonn) -> .cb;
        node.cafes(area.frankfurt) -> .cf;
        (.cb; .cb;) -> .cfb;
        out .cfb;
    \end{lstlisting}\end{lstwrap}

    Same query, but flattened:
    \begin{lstwrap}\begin{lstlisting}
        area[name="Bonn"] -> .bonn;
        area[name="Frankfurt"] -> .frankfurt;
        ( node[amenity="cafe"](area.bonn); node[amenity="cafe"](area.frankfurt); );
        out;
    \end{lstlisting}\end{lstwrap}
\end{example}

\subsubsection{And everything else}
There are more available statements within Overpass, including:
\begin{itemize}
    \item Recurse up/down relations - get items that are within relations in the
        current set and get relations the items in the input set are contained
        in
    \item ``is in'' - get areas the items in the input set are contained in
    \item Recurse along ways - get next, previous objects along way
    \item Fixed point - perform a block on an input set until it returns the
        same set
\end{itemize}

Since they haven't been used in this reference implementation, they will not be
discussed in detail. For extra information, see \cite{overpassql}.
\end{document}
