\documentclass[main.tex]{subfiles}
\begin{document}
\subsection{Natural language parsing}
This section deals with the part of the pipeline which transforms the input
query (raw text) into a payload (when targeting Minipass, the payload is a
Minipass term).

The process can be broken down into 3 stages:
\begin{itemize}
    \item Tokenisation: Generating a sequence of tokens from the input text.
        This stage is natural-language dependant at source-code level (targeting
        a different natural language requires attaching a different tokeniser).
        The implementation comes with a simple English tokeniser made with the
        help of external libraries.
    \item CCG parsing: Applying the rules specified in the input grammar (which
        comes as a \code{.ccg} file) and finding valid parse trees for the
        input query (if any).
    \item Term generation: generating final terms by composing the subterms
        within parse tree leaves
\end{itemize}

\subsubsection{Tokenisation}\label{sec:tokenisation}
Tokens are represented as sets of key-value pairs\footnote{there is no
    restriction on the uniqueness of keys}. Currently, there is one
implemented tokeniser wrapper, which will be described in this section. Other
tokenisers (for different natural languages, for example) may easily be added
due to the project's modular architecture.

The English tokeniser wrapper uses a simple tokeniser from the \code{NLP.POS}
package from the \code{chatter} Haskell library \cite{chatter}.
After splitting the input text into distinct tokens, a POS tagger from the
same library is used to assign a part of speech to each token. The POS
tagger uses the ``averaged perceptron'' model \cite{collins} and is trained
on the Penn Treebank POS tag corpus \cite{penn}.

The raw token string is included under key \code{raw} in the token object and
the POS tag is included under key \code{pos}.

After POS tagging, tokens are subjected to lemmatisation. This is done using
WordNet's \code{morph} library \cite{wordnet}. Since at the time of writing
this thesis available Haskell bindings to WordNet either included no \code{morph}
support or couldn't run with the most recent version of GHC, a small binding
library which wraps the C interface provided with WordNet was written. The
\code{morph} lemmatiser is given a token and its POS tag and returns the lemma
for this token. The lemma (if exists\footnote{Tokens like punctuation items have
    no lemmas}) is then included under key \code{lemma} in the token object.

The tokeniser also emits special tokens for ``begin of sentence'' and ``end of
sentence''. These are included as special \code{mark=begin} and \code{mark=end}
tokens objects.

\begin{example}
    The following text:
    \begin{center}
        \code{Suddenly there came a tapping.}
        \code{As if someone gently rapping.}
    \end{center}

    Produces the following token list (one token on each row):
\begin{lstwrap}\begin{lstlisting}
mark=begin
            raw=Suddenly    pos=rb      lemma=suddenly
            raw=there       pos=ex      lemma=there
            raw=came        pos=vbd     lemma=come
            raw=a           pos=dt      lemma=a
            raw=tapping     pos=nn      lemma=tapping
            raw=.           pos=.       lemma=.
mark=end
mark=begin
            raw=As          pos=in      lemma=as
            raw=if          pos=in      lemma=if
            raw=someone     pos=nn      lemma=someone
            raw=gently      pos=rb      lemma=gently
            raw=rapping     pos=vbg     lemma=rap
            raw=.           pos=.       lemma=.
mark=end
\end{lstlisting}\end{lstwrap}
\end{example}

\subsubsection{CCG parsing}
After tokenising, all rule matchers (\cref{sec:ccglang}) are evaluated against
each token one by one\footnote{This naïve quadratic approach may be replaced
    by a decision tree-based traversal of the set of matchers with little
    effort.} and a list of (Category, Term) pairs is generated for each token.
Then, proceeding with the categorial CYK algorithm, a set of items (in the
sense of \cref{sec:cyk}) is constructed recursively.

The maximum composition arity $n$ in the implementation is set to $1$ since no need
for more arguments has arisen. If the grammar demands a higher arity, this
is easily remedied.

The straight-forward recursive implementation of the algorithm relies on a
memoiser in order to avoid recomputing shared items. Memoisation is used in
several places throughout the code, and two generic memoisers have been
implemented:
\begin{itemize}
    \item A lazy memoiser, based on an infinite bit trie which stores the return
        value for each possible argument. Only a the already accessed paths are
        stored in memory at a given moment, while the rest of the possible
        subtrees are represented by Haskell thunks, which are computed on
        demand.
    \item A pragmatic memoiser which stores computed values in a hashmap which
        is accessed via \code{unsafePerformIO}.
\end{itemize}

The lazy memoiser, while elegant, is about twice as slow as the pragmatic
memoiser, which is used by default.

Another performance consideration is the fact that when combining categories,
care is taken to share common parts of categories with ones constructed from
them\footnote{This comes for free because of Haskell's immutability mechanism:
    the only care that needs to be taken is not obstructing the compiler in
    that regard.
}.

\subsubsection{Parse forests and generating terms}\label{sec:termgen}
Since there are potentially an exponential number of derivation trees for a
single query, many of them sharing an identical trunk due to the way
the categorial CYK algorithm works. Thus, it is beneficial to encode the resulting
parse forest in such a way as to share the common trunks between trees, and
also be able to avoid enumerating all of them when performing simple manipulations.

This is a simplified definition of the \code{ParseTree} and \code{ParseForest}
datatypes used in the implementation. Here, \code{cat} is a category,
\code{payload} is a payload attached to the matched token (a $\lambda$-term and
metadata about the token itself), and \code{(CombineRule cat)} is a rule\footnote{
    The syntax ``\code{(CombineRule cat)}'' relies on the \code{TypeFamilies}
    Haskell extension. Type family constraints have been omitted for brevity.
} by which this derivation was produced (used for debugging and visualisation):
\begin{lstwrap}\begin{lstlisting}
    data ParseTree cat payload where
        Leaf cat payload
        Vert cat (CombineRule cat) (ParseTree cat payload) (ParseTree cat payload)

    data ParseForest cat payload
        = ParseForest cat [MultiNode cat payload]

    data MultiNode cat payload where
        MultiLeaf payload           (MultiNode cat payload)
        MultiVert (CombineRule cat) (ParseForest cat payload) (ParseForest cat payload)
\end{lstlisting}\end{lstwrap}

A \code{ParseForest} is recursively generated from the set of items produced
by the parsing algorithm, starting from the initial category and the whole query.

From the \code{ParseForest}, all \code{ParseTree}s are enumerated in a lazy
list: this way most of the encoded trees will never be evaluated individually
(unless requested by the user).

A $\lambda$-term is generated for each relevant tree, following the procedure
from \cref{def:semfn}. Afterwards, in the case of Minipass terms, their types are
squashed\footnote{See \cref{def:squashfn}.} into the bare Minipass type system,
discarding any type information that was useful only for parsing.
\end{document}
