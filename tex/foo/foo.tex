 % !TEX program = xelatex
\documentclass[12pt]{extarticle}
\usepackage[T1]{fontenc}
\usepackage{fontspec}

\defaultfontfeatures{Ligatures=TeX}
\setmainfont{Noto Serif}
\setsansfont{Noto Sans}
\setmonofont{Noto Sans Mono}

\usepackage[bulgarian,english]{babel}
\usepackage{indentfirst}
\usepackage[a4paper, portrait, margin = 2.5 cm]{geometry}
\usepackage{url}
\usepackage{color}
\usepackage{xcolor}
\usepackage{graphicx}
\usepackage{adjustbox}
\usepackage{listings}
\usepackage{syntax}
\renewcommand{\baselinestretch}{1.1}
\setlength{\emergencystretch}{3em}
\setlength{\parindent}{0pt}


\lstset{
	backgroundcolor = \color{light-gray},
    language = C,
    xleftmargin = 1cm,
    framexleftmargin = 1em,
    basicstyle=\ttfamily,
	moredelim=[is][\underbar]{_}{_},
}

\usepackage{color}
\definecolor{Bluish}{rgb}{0.39,0.55,0.78}
\definecolor{light-gray}{gray}{0.9}

\usepackage{hyperref}
\hypersetup{
    colorlinks=true,
    linktoc=all,
    citecolor=black,
    filecolor=black,
    linkcolor=black,
    urlcolor=black
}

\usepackage{tabularx}


\setlength{\grammarparsep}{20pt plus 1pt minus 1pt} % increase separation between rules
\setlength{\grammarindent}{12em} % increase separation between LHS/RHS


\title{ble ble}
\author{Ангел Ангелов <dexterlb@qtrp.org>}
\date{2018}

\begin{document}
% uncomment for fancy title:
% \maketitle
% \thispagestyle{empty}
% \tableofcontents
% \pagebreak

\section{The Intermediate Language}
Note: this is very much subject to change.

Let our world be a directed graph whose vertices we call \emph{elements}.
Each \emph{edge} carries a \emph{label}. Each \emph{label} is a polymorphic
\emph{List} of \emph{Numbers}, \emph{Strings} and other \emph{Lists}.

\begin{grammar}

<term> ::= <variable>
\alt <constant>
\alt <term> <term>
\alt `lambda' <declaration> `{' <term> `}'
\alt `(' <term> `)'

<constant> ::= <stringliteral>
          \alt <numberliteral>
          \alt `and'
          \alt `or'
          \alt `get'
          \alt `next'
          \alt `empty'
          \alt `consNum'
          \alt `consString'
          \alt `consList'


<variable> ::= <identifier>

<declaration> ::= <variable> `:' <type>
\alt <variable>

<type> :: <type> `->' <type>
\alt <basictype>
\alt `(' <type> `)'

<basictype> ::= `Set'
\alt `Num'
\alt `String'
\alt `List'

<identifier> ::= [a-zA-Z_][a-zA-Z_0-9]*

<numberliteral>     ::= -?[0-9]+(\textbackslash.[0-9]+)?

<stringliteral>     ::= <todo>
\end{grammar}

For ease of writing, let us consider that types are right-associative,
applications are left-associative and that we can write list literals
within square brackets:
\begin{lstlisting}
[42, "foo"] := consNum 42 (consString "foo" empty)
\end{lstlisting}

And here are the types for all constants:

\begin{lstlisting}
<stringliteral> : String
<numberliteral> : Num

and       : Set -> Set -> Set   -- set intersection
or        : Set -> Set -> Set   -- set union
not       : Set -> Set          -- set negation

get       : List -> Set         -- get elements by label
next      : List -> Set -> Set  -- traverse edges

-- constructing lists
empty       :                     List
consNum     : Num      -> List -> List
consString  : String   -> List -> List
consList    : List     -> List -> List

\end{lstlisting}

With this abstraction, we can represent most of the Overpass queries for
constructing sets (filters, recurse, in/is_in, set operations, conditions)
by encoding them as labels. Here are some examples:

\begin{lstlisting}
within : Num -> Set -> Set
    := lambda dist { next ["around", dist] }

onStreet : String -> Set
    := lambda str  { next [
        "tagFilter", ["==", "addr:street", str]] }
\end{lstlisting}

\end{document}

