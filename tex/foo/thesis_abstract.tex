\documentclass[11pt,a4paper]{article}
\usepackage[utf8]{inputenc}
\usepackage[bulgarian]{babel}

\title{Категорийни граматики за геопространствени заявки}

\begin{document}
\maketitle

\section*{Анотация}

Категорийните граматики са вид формални граматики, които представят синтактичните елементи с помощта на категории: базови множества и формални функции над тях. Основавайки се на принципа за композиционалност, който гласи, че семантиката на текст е композиция от семантиките на съставните му части, инструментариумът на $\lambda$-смятането е приложим за автоматично транслиране на изречение на естествен език до изрази на формален език със строго дефинирана семантика (напр. аритметични изрази, формули, програмни оператори).

Категорийните граматики биха могли да бъдат използвани като помощно средство за интерпретирането на команди на естествен език чрез превеждането им до подходящ формален език за заявки, например в контекста на виртуални асистенти, предоставящи гласов потребителски интерфейс. Въпросите, свързани с търсенето на географски обекти са един от популярните видове потребителски заявки. Изпълнението им е свързано с работа с географски данни със разнородна структура, често описвани чрез йерархия от множества от обекти с различни релации между тях. Наличието на отворени географски данни като OpenStreetMap (OSM) правят възможно експериментирането с геопространствени заявки за търсене на реални обекти. Предоставеният от OSM език за заявки Overpass е със сложен синтаксис и не е достъпен за работа от потребители без технически познания. Категорийните граматики изглеждат като подходящ кандидат за предоставяне на удобен интерфейс за взаимодействие със географска база от данни чрез автоматичен превод на заявки от естествен към формален език.

\section*{Цел на дипломната работа}
Да се изследва дали може да се дефинират подходящи категорийни граматики, които позволяват ефикасен анализ на геопространствени заявки на подмножество на английския език и превеждането им до заявки на формален език за геопространствени заявки.

\section*{Задачи на дипломната работа}

\begin{enumerate}
\item Мотивиран избор на подходящ тип категорийна граматика
\item Мотивиран избор на подходящо подмножество на английски език за формулиране на геопространствени заявки
\item Дефиниция на минимален междинен език за геопространствени заявки, който ще бъде целеви език на категорийните граматики и ще бъде ефективно преводим до стандартен език за геопространствени заявки (напр. Overpass)
\item Дефиниция на една или няколко категорийни граматики за анализ на изречения на естествен език от избраното подмножество
\item Дефиниция на подходящи типови системи за категорийните граматики
\item Алгоритъм за синтактичен анализ на изречение на естествен език съгласно дефинираните категорийни граматики
\item Програмна реализация на алгоритъма за синтактичен анализ
\item Доказателство за тотална коректност на предложения алгоритъма за синтактичен анализ
\item Анализ на сложността на предложения алгоритъм за синтактичен анализ
\item Алгоритми за превод от дървото на извод до междинния език за геопространствени заявки
\item Доказателство за тоталност на алгоритъма за превод
\item Анализ на сложността на предложения алгоритъм за превод
\item Програмна реализация на транслатор от изречение на естествен език до междинния език за заявки чрез семантичен извод за избраните категорийни граматики
\item Програмна реализация на транслатор от израз на междинния език до стандартен език за геопространствени заявки
\end{enumerate}
\end{document}
