\documentclass[main.tex]{subfiles}
\begin{document}
\subsection{Simply typed $\lambda$-calculus}
\label{sec:purelambda}

For representing compositional terms, a version of simply typed lambda calculus
will be used.

\subsubsection{Types}
\label{purelambda:types}
Let $\mathbb{T}$ be a magical set of items we call
    \emph{type symbols}.

\begin{defn}
    For a set of type symbols $T$, its \emph{type closure}
    $\tclos{T} \subseteq (T \, \dot\cup \,\{\, \tot, ), ( \,\})^*$ is defined
    inductively:

    \begin{itemize}
        \item $\sigma \in T \implies \sigma \in \tclos{T}$
        \item $\sigma, \tau \in \tclos{T} \implies (\sigma \tot \tau) \in \tclos{T}$
    \end{itemize}
\end{defn}

\begin{defn}
    The function $ts : \tclos{\mathbb{T}} \rightarrow 2^\mathbb{T}$, which
    returns the set of type symbols used in a complex type, is defined inductively:

    \begin{itemize}
        \item if $\sigma \in \mathbb{T}$ then $ts(\sigma) = \sigma$
        \item if $\sigma = \sigma' \rightarrow \sigma''$ then
            $ts(\sigma) = ts(\sigma') \cup ts(\sigma'')$
    \end{itemize}
\end{defn}

We will now introduce a notation for easily getting the types of items -
$\typeof{x}$ shall mean "the type of $x$":

\begin{defn}
    We call a set $X$ \emph{typed in} $T$ $\iff$ there is a function
    \[ \typeoff : X \rightarrow \tclos{T} \]
    which we call a \emph{typing function}.
\end{defn}

\begin{defn}
    If $X, Y$ are sets of type symbols and $\varphi: X \rightarrow \tclos{Y}$,
    then $\hat{\varphi} : \tclos{X} \rightarrow \tclos{Y}$ is defined inductively:
    \begin{equation*}
        \hat{\varphi}(\sigma) =
        \begin{cases*}
            \varphi(\sigma), & $\sigma \in X$ \\
            \hat{\varphi}(\sigma') \rightarrow \hat{\varphi}(\sigma''), &
                $\sigma = \sigma' \rightarrow \sigma'' \in \tclos{X}$ \\
        \end{cases*}
    \end{equation*}
\end{defn}

\subsubsection{Terms}
To define $\lambda$ terms with the types introduced above, a system similar
to Church's $\lambda_\rightarrow$ \cite[chap.~2.4]{ttfp} shall be used. It
extends the original system with constants.

Let $\mathbb{V}$ be an magical set. We shall call its
elements \emph{variables}.
Respectively, let $Const$ be a magical set whose elements
we call \emph{constants}.

\begin{defn}
    Let $C$ be a set of constants, typed in $T$.

    We can define the set of
    \emph{pre-typed $\lambda$-terms} ($\Lambda_T^C$).

    \begin{itemize}
        \item Constant:    \[ c \in C \implies c \in \Lambda_T^C \]
        \item Variable:    \[ v \in \mathbb{V} \implies v \in \Lambda_T^C \]
        \item Application: \[ A, B \in \Lambda_T^C \implies (AB) \in \Lambda_T^C \]
        \item Abstraction: \[ v \in \mathbb{V}, A \in \Lambda_T^C, \sigma \in \tclos{T}
                \implies (\lambda v : \sigma \abstr A) \in \Lambda_T^C \]
    \end{itemize}
\end{defn}

\begin{defn}
    Statement, declaration, context, judgement \cite[chap.~2.4]{ttfp}
    \begin{itemize}
        \item For $M \in \Lambda_T^C, \sigma \in \tclos{T}$, $M : \sigma$ is called
            a \emph{statement}. $M$ is called a \emph{subject} and $\sigma$
            is called a \emph{type}.
        \item A statement with a variable as a subject is called a \emph{declaration}.
        \item A set of declarations with different subjects is called a \emph{context}.
        \item A \emph{judgement} has the form $\Gamma \vdash M: \sigma$, where
            $\Gamma$ is a context and $M: \sigma$ is a statement.
    \end{itemize}

    Moreover, appending to contexts is defined as follows:
    \[ \Gamma \circ x : \sigma = \{ y : \tau \in \Gamma \mid y \neq x \}
       \cup \{ x : \sigma \} \]
\end{defn}

\begin{defn}
    To define what it means for a term $M \in \Lambda_T^C$ to have a type,
    we will use derivation rules.
    \begin{itemize}
        \item Constant
            \cenderiv{
                \hypo{c \in C}
                \infer1{\Gamma \vdash c : \typeof{c}}
            }
        \item Variable
            \cenderiv{
                \hypo{x : \sigma \in \Gamma}
                \infer1{\Gamma \vdash x : \sigma}
            }
        \item Application
            \cenderiv{
                \hypo{\Gamma \vdash A : \sigma' \tot \sigma''}
                \hypo{\Gamma \vdash B : \sigma'}
                \infer2{\Gamma \vdash (AB) : \sigma''}
            }
        \item Abstraction
            \cenderiv{
                \hypo{\Gamma \circ x : \tau \vdash A : \sigma}
                \infer1{\Gamma \vdash (\lambda x : \tau \abstr A) : \tau \tot \sigma}
            }
    \end{itemize}

    Now, $\typeof{M}$ for $M \in \Lambda_T^C$ is defined as such:
    \[
        \typeof{M} =
        \begin{cases*}
            \sigma & if $\varnothing \vdash M : \sigma$ \\
            \neg ! & otherwise
        \end{cases*}
    \]
    Of course, the above function is nowhere defined for non-closed terms.
\end{defn}

\end{document}
