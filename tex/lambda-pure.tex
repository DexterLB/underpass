\documentclass[main.tex]{subfiles}
\begin{document}
\subsection{Simply typed $\lambda$-calculus}
\label{sec:purelambda}

For representing compositional terms, a version of the simply typed lambda calculus
will be used.

\subsubsection{Types}
\label{purelambda:types}
Let $\tsymbs \defeq \{ t', t'', t''' ... \}$ be a countable set of symbols that we will call
    \emph{type symbols}\footnote{See \cref{sec:magic}}.

\begin{defn}
    \label{def:tclos}
    For a set of type symbols $T$, its \emph{type closure}
    $\tclos{T} \subseteq (T \, \cup \,\{\, \tot, ), ( \,\})^*$ is defined
    inductively:

    \begin{itemize}
        \item $\sigma \in T \implies \sigma \in \tclos{T}$
        \item $\sigma, \tau \in \tclos{T} \implies (\sigma \tot \tau) \in \tclos{T}$
    \end{itemize}
\end{defn}

\begin{convention}
    For clarity, braces shall be omitted in the notation for types by regarding
    the $\tot$ operation as right-associative:
    \[ \sigma \tot \tau \tot \eta \]
    will mean
    \[ ( \sigma \tot ( \tau \tot \eta ) ) \]
\end{convention}

\begin{defn}
    The function $\tss : \tclos{\tsymbs} \fun 2^\tsymbs$, which
    returns the set of type symbols used in a complex type, is defined inductively:

    \begin{itemize}
        \item $\sigma \in \tsymbs \implies \ts{\sigma} \defeq \sigma$
        \item $\sigma = \sigma' \tot \sigma'' \implies
            \ts{\sigma} \defeq \ts{\sigma'} \cup \ts{\sigma''}$
    \end{itemize}
\end{defn}

We will now introduce a notation for easily obtaining the types of items ---
$\typeof{x}$ shall mean ``the type of $x$'':

\begin{defn}
    We call a set $X$ \emph{typed in} $T$ when there is a function\fixednote{$\tot$ се използва в един и същ контекст с две различни значения}
    \[ \typeoff : X \fun \tclos{T} \]
    that we call a \emph{typing function}.
\end{defn}

\begin{defn}
    If $X, Y$ are sets of type symbols and $\varphi: X \fun \tclos{Y}$,
    then $\hat{\varphi} : \tclos{X} \fun \tclos{Y}$ is defined inductively:
    \begin{equation*}
        \hat{\varphi}(\sigma) =
        \begin{cases*}
            \varphi(\sigma) ,& $\sigma \in X$ \\
            \hat{\varphi}(\sigma') \tot \hat{\varphi}(\sigma'') ,&
                $\sigma = \sigma' \tot \sigma'' \in \tclos{X}$ \\
        \end{cases*}
    \end{equation*}
\end{defn}

\subsubsection{Terms}
To define $\lambda$ terms with the types introduced above, a system similar
to Church's $\lambda_\tot$ \cite[chap.~2.4]{ttfp} shall be used. It
extends the original system with constants.

Let $\lvars = \{ \alpha', \alpha'', \alpha''' ... \}$ be a countable set of symbols
that we will call \emph{variables} and $\const = \{ c', c'', c''' ... \}$ be a countable
set of symbols that we will call \emph{constants}.

We can now define the set of
\emph{pre-typed $\lambda$-terms with constants $C$ typed in $T$} ($\plambda{T}{C}$).

\begin{defn}
    \label{def:lambdaterm}
    Let $C \subset \const$ be a set of constants, typed in $T$.

    $\plambda{T}{C}$ is defined inductively:
    \begin{itemize}
        \item Constant:    \[ c \in C \implies c \in \plambda{T}{C} \]
        \item Variable:    \[ v \in \lvars \implies v \in \plambda{T}{C} \]
        \item Application: \[ A, B \in \plambda{T}{C} \implies (AB) \in \plambda{T}{C} \]
        \item Abstraction: \[ v \in \lvars, A \in \plambda{T}{C}, \sigma \in \tclos{T}
                \implies (\lambda v : \sigma \abstr A) \in \plambda{T}{C} \]
    \end{itemize}
\end{defn}

\begin{convention}
    In this text, chained abstractions will be squashed, i.e.
    $\lambda x_1 : T_1, x_2: T_2, ..., y_n: T_n \abstr M$ shall mean
    $\lambda x_1 : T_1 \abstr (\lambda x_2: T_2 \abstr (... (\lambda x_n: T_n \abstr M) ...))$.
\end{convention}

We will differentiate between \emph{free variables} and \emph{bound variables}
in a term. Intuitively, bound variables are ``bound'' by abstraction, while
free variables are not.

We will now formally define the function $\fvv$ that gives us all free variables of a
term.
\begin{defn}
    Let $\fvv : \plambda{T}{C} \fun 2^{\lvars}$ be defined as:
    \[
        \fv{M} =
        \begin{cases*}
            \emptyset ,& $M = c \in C$ \\
            \{ v \} ,& $M = v \in \lvars$ \\
            \fv{A} \cup \fv{B} ,& $M = (AB)$ \\
            \fv{A} \setminus \{ v \} ,& $M = (\lambda v : \sigma \abstr A)$ \\
        \end{cases*}
    \]
\end{defn}

\begin{defn}
    \label{def:context}
    Statement, declaration, context, judgement \cite[chap.~2.4]{ttfp}
    \begin{itemize}
        \item If $M \in \plambda{T}{C}, \sigma \in \tclos{T}$, then $M : \sigma$ is called
            a \emph{statement}. $M$ is called a \emph{subject} and $\sigma$
            is called a \emph{type}.
        \item A statement with a variable as a subject is called a \emph{declaration}.
        \item A set of declarations with different subjects is called a \emph{context}.
        \item A \emph{judgement} has the form $\Gamma \vdash M: \sigma$, where
            $\Gamma$ is a context and $M: \sigma$ is a statement.
    \end{itemize}

    Moreover, appending to contexts is defined as follows:
    \[ \Gamma \circ x : \sigma = \{ y : \tau \in \Gamma \mid y \neq x \}
       \cup \{ x : \sigma \} \]
\end{defn}

The $\circ$ operation shall be regarded as left-associative throughout this
text.

To define what it means for a term $M \in \plambda{T}{C}$ to have a type,
we will use derivation rules.
\begin{defn}
    \label{def:termderiv}
    Derivation rules for typed $\lambda$-terms:
    \begin{itemize}
        \item Constant
            \cenderiv{
                \hypo{c \in C}
                \infer1{\Gamma \vdash c : \typeof{c}}
            }
        \item Variable
            \cenderiv{
                \hypo{x : \sigma \in \Gamma}
                \infer1{\Gamma \vdash x : \sigma}
            }
        \item Application
            \cenderiv{
                \hypo{\Gamma \vdash A : \sigma' \tot \sigma''}
                \hypo{\Gamma \vdash B : \sigma'}
                \infer2{\Gamma \vdash (AB) : \sigma''}
            }
        \item Abstraction
            \cenderiv{
                \hypo{\Gamma \circ x : \tau \vdash A : \sigma}
                \infer1{\Gamma \vdash (\lambda x : \tau \abstr A) : \tau \tot \sigma}
            }
    \end{itemize}
\end{defn}

\begin{lemma}\label{lemma:inversion}
    \emph{Inversion lemma}:
    \begin{itemize}
        \item If $\Gamma \vdash c : \sigma$ for $c \in C$, then $\sigma = \typeof{c}$
        \item If $\Gamma \vdash x : \sigma$ for $x \in \lvars$, then $x : \sigma \in \Gamma$
        \item If $\Gamma \vdash (AB) : \sigma''$, then for some $\sigma'$:
            \begin{itemize}
                \item $\Gamma \vdash A : \sigma' \tot \sigma''$
                \item $\Gamma \vdash B : \sigma'$
            \end{itemize}
        \item If $\Gamma \vdash (\lambda x : \tau \abstr A) : \nu$, then
            $\nu = \tau \tot \sigma$ and $\Gamma \circ x : \tau \vdash A : \sigma$.
    \end{itemize}
\end{lemma}
\begin{proof}
    It follows immediately from the \cref{def:termderiv}, since the cases are
    non-overlapping.
\end{proof}
Now, we will prove that every well-typed term has a unique type:\fixednote{формално погледнато, преди това трябва да докажеш Лема за обръщането, иначе с индукция по $M$ не можеш да правиш заключения зи $\vdash$}
\begin{prop}
    \label{prop:uniquetypespure}
    \[ \Gamma \vdash M : \sigma' \land \Gamma \vdash M : \sigma''
        \implies \sigma' = \sigma''. \]
\end{prop}
\begin{proof}
    We will assume that $\Gamma \vdash M : \sigma'$ and $\Gamma \vdash M : \sigma''$
    and prove $\sigma' = \sigma''$ by induction over the construction of $M$.
    In every case, we will make use of \cref{lemma:inversion} to decompose
    the right-hand side of $\vdash$.

    \begin{itemize}
        \item $M = c \in C$:

            $\sigma' = \typeof{c} = \sigma''$

        \item $M = x$ for some $x : \sigma' \in \Gamma$ and $x : \sigma'' \in \Gamma$:

            Since $\Gamma$ is a context (e.g. each subject appears only once),
            there is only one declaration with subject $x$, thus $\sigma' = \sigma''$.

        \item $M = (AB)$:

            We have $\Gamma \vdash (AB) : \sigma'$ and $\Gamma \vdash (AB) : \sigma''$.
            This means that
            \begin{align*}
                \Gamma \vdash& A : \eta' \tot \sigma'\\
                \Gamma \vdash& B : \eta'
            \end{align*}
            and
            \begin{align*}
                \Gamma \vdash& A : \eta'' \tot \sigma''\\
                \Gamma \vdash& B : \eta''
            \end{align*}
            But inductively $A$ has a unique type, thus $\eta' \tot \sigma'
            = \eta'' \tot \sigma''$, therefore $\sigma' = \sigma''$.

        \item $M = (\lambda x : \tau \abstr A)$, $\sigma' = \tau \tot \eta'$
            and $\sigma'' = \tau \tot \eta''$.

            This means that $\Gamma \circ x : \tau \vdash A : \eta'$ and
            $\Gamma \circ x : \tau \vdash A : \eta''$. Inductively, $\eta'
            = \eta''$: therefore, $\sigma' = (\tau \tot \eta') = (\tau \tot \eta'')
            = \sigma''$.
    \end{itemize}
\end{proof}

Since we proved that types for terms are unique, we can now define a typing
function (which will work on well-typed closed terms):
\begin{defn}
    Now, $\typeof{M}$ for $M \in \plambda{T}{C}$ is defined as follows:
    \[
        \typeof{M} =
        \begin{cases*}
            \sigma, & if $\emptyset \vdash M : \sigma$, \\
            \neg !, & otherwise.
        \end{cases*}
    \]
\end{defn}

\end{document}
