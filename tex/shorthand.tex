\documentclass[main.tex]{subfiles}
\begin{document}

\subsubsection{Composing phrases}
The CCG formalism we have established allows a fixed set of terminals,
which is usually chosen to be the set of all \emph{tokens}, or words,
in the target natural language. This works fairly well, because it is
convenient to write rules at word level.

Sometimes, however, we come across fixed phrases, which come as several
adjacent terminals but we want to regard as a single terminal.

\begin{example}
    Suppose we have a $\lambda$-term $parkinglot$ with type $Amenity$.
    
    For example, we might want to assign the category $Amenity$ with term
    $parkinglot$
    to the query \code{parking lot} which is composed of two terminals.
    
    One way to do this is to set dummy categories to all but one of the
    terminals and have the main terminal accept the dummy categories as
    arguments:
   
    \begin{itemize}
        \item Add a new type/category $Dummy$.
        \item Choose a closed term $M$ of type $Dummy$.
        \item GSet
            \begin{align*}
                Amenity \rci{\modstar} Dummy &\in f(parking)\\
                Dummy &\in f(lot) \\
                \\
                \lambda d: Dummy \abstr parkinglot &\in \varphi(parking, Amenity \rci{\modstar} Dummy)  \\
                M &\in \varphi(lot, Dummy) \\
            \end{align*}
    \end{itemize}
    
    This approach yields a derivation subtree for \code{parking lot} which
    has the correct category and semantics:
    \centree{.{$Amenity$}
        [ .{$Amenity \rci{\modstar} Dummy$} [ .{$parking$} ] ]
        [ .{$Dummy$} [ .{$lot$} ] ]
    }
\end{example}

\begin{convention}
    \label{hack:phrases}

    While defining a CCG $G = \langle \Sigma, N, S, f, n \rangle$ such that
    $N$ is typed in $T$,
    the following assertions:
    \begin{align*}
        X &\in f[\alpha_1 \alpha_2 ... \alpha_k] \\
        M &\in \varphi[\alpha_1 \alpha_2 ... \alpha_k, X] \\
    \end{align*}
    where $X \in \cclos{N}$ and $M$ is a $\lambda$-term typed in $T$, have
    the following semantics:
    \begin{itemize}
        \item $\sigma_2, \sigma_3 ... \sigma_k \in T$, where $\sigma_i$ is a new type
            whose choice\footnote{
                When using $\lambda$-terms which allow subtypes and a wildcard type,
                such as the type system used in Minipass, it is
                convenient to create new types by subtyping the identity type
                $\wildcard \tot \wildcard$ and using
                $(\lambda x: \wildcard \abstr x)$ as a dummy term.
            }
            depends on $\alpha_1, ... \alpha_k$ and $i$.
        \item $X_2, X_3, ..., X_k \in N$, where $X_i$ is a new category
            such that $\typeof{X_i} = \sigma_i$ and whose choice
            depends on $\sigma_i$.
        \item $M_2, M_3, ..., M_k$ are arbitrary $\lambda$-terms whose types are
            respectively $\sigma_2, \sigma_3, ..., \sigma_k$
        \item New added categories to $f$:
            \begin{itemize}
                \item $X \rci{\modstar} X_2 \rci{\modstar} X_3
                    \rci{\modstar} ... \rci{\modstar} X_k \in f(\alpha_1)$
                \item For each $i \in \{ 2 ... k \}, X_i \in f(\alpha_i)$
            \end{itemize}
        \item New added terms to $\varphi$:
            \begin{itemize}
                \item $(\lambda x_2: \sigma_2, x_3: \sigma_3, ..., x_k: \sigma_k \abstr
                    M) \in \varphi(\alpha_1, X \rci{\modstar} X_2 \rci{\modstar} X_3
                    \rci{\modstar} ... \rci{\modstar} X_k)$
                \item For each $i \in \{ 2 ... k \}, M_i \in \varphi(\alpha_i, X_i)$
            \end{itemize}
    \end{itemize}
\end{convention}

\subsubsection{Shorthand for grammar specification}
\label{shorthand}
When manually defining grammars, it is tedious to write definitions
for $f$ and $\psi$ using standard mathematical notation.

Thus, we will adopt a special syntax, assuming a CCG
$G = \langle \Sigma, N, S, f, n \rangle$ with an arbitrarily large $n$.

The following assertion:
\gramshort{
    \gramrow{\alpha}{X_{1}}{M_{1}}
    \gramrow{}{X_{2}}{M_{2}}
    &  & ... & \\
    \gramrow{}{X_m}{M_m}
}
shall mean that:
\begin{align*}
    \{ X_1, X_2 ... X_m \} & \subseteq f[\alpha] \\
    M_1 & \in \psi[\alpha, X_1] \\
    M_2 & \in \psi[\alpha, X_2] \\
    ... & \\
    M_m & \in \psi[\alpha, X_m] \\
\end{align*}

Here, the square brackets denote possible phrase matching (\cref{hack:phrases}). It can
be ignored or swept under the rug, which makes square brackets behave like
round brackets.

\begin{example}
    This meta-example shows a couple of rules to illustrate the syntax:
    \gramshort{
        \gramrow{near}{ GSet \lc GSet \rc GSet }{ \lambda from: GSet, things: GSet \abstr }
            \gramrow{}{}{ and \app things \app (within \app nearDistance \app from) }
        \gramrow{german \sq city}{ GSet }{ and \app inGermany \app city }
        \gramrow{}{ GSet \rc Name }{ \lambda n{:} Name \abstr and \app (named \app n) \app (and \app inGermany \app city) }
    }
    The grammar it specifies is the following:
    \begin{align*}
        f[german \sq city] &= \{ GSet, GSet \rc Name \} \\
        f(near) &= \{ GSet \lc GSet \rc GSet \} \\
        \\
        \psi(near, GSet \lc GSet \rc GSet) &= \lambda from: GSet, things: GSet \abstr \\
        & \phantom{=} \qquad and \app things \app (within \app nearDistance \app from) \\
        \psi[german \sq city, GSet] &= and \app inGermany \app city \\
        \psi[german \sq city, GSet \rc GSet] &= \lambda n{:} Name \abstr and \app (named \app n) \app (and \app inGermany \app city) \\
     \end{align*}

     Here we assign two different categories with their respective semantics
     to the phrase \code{german city} - one which can be seen in the query
     \code{"fountains in a german city"} and another in the query
     \code{"the german city Frankfurt"}.
 \end{example}

Also, most examples from here on will assume that grammars have a starting
category $GSet$, to be as close as possible to the implementation in
\cref{sec:minipass}.
\end{document}
