\documentclass[main.tex]{subfiles}
\begin{document}
\subsubsection{Parsing phrases}
\fixme{write this}

\subsubsection{Shorthand for specifying grammars}
\label{shorthand}
When manually defining grammars, it is tedious to write definitions
for $f$ and $\psi$ using standard mathematical notation.

Thus, we will adopt a special syntax, assuming a CCG
$G = \langle \Sigma, N, S, f, n \rangle$ with an arbitrarily large $n$.

The following assertion:
\gramshort{
    \gramrow{\alpha}{X_{1}}{M_{1}}
    \gramrow{}{X_{2}}{M_{2}}
    &  & ... & \\
    \gramrow{}{X_m}{M_m}
}
shall mean that:
\begin{align*}
    \{ X_1, X_2 ... X_m \} & \subseteq f[\alpha] \\
    M_1 & \in \psi[\alpha, X_1] \\
    M_2 & \in \psi[\alpha, X_2] \\
    ... & \\
    M_m & \in \psi[\alpha, X_m] \\
\end{align*}

Here, the square brackets denote possible phrase matching (\cref{hack:phrases}). It can
be ignored or swept under the rug, which makes square brackets behave like
round brackets.

\example{
    This meta-example shows a couple of rules to illustrate the syntax:
    \gramshort{
        \gramrow{near}{ Set \lc Set \rc Set }{ \lambda from: Set, things: Set \abstr }
            \gramrow{}{}{ and \app things \app (within \app nearDistance \app from) }
        \gramrow{german \sq city}{ Set }{ and \app inGermany \app city }
        \gramrow{}{ Set \rc Name }{ \lambda n{:} Name \abstr and \app (named \app n) \app (and \app inGermany \app city) }
    }
    The grammar it specifies is the following:
    \begin{align*}
        f[german \sq city] &= \{ Set, Set \rc Name \} \\
        f(near) &= \{ Set \lc Set \rc Set \} \\
        \\
        \psi(near, Set \lc Set \rc Set) &= \lambda from: Set, things: Set \abstr \\
        & \phantom{=} \qquad and \app things \app (within \app nearDistance \app from) \\
        \psi[german \sq city, Set] &= and \app inGermany \app city \\
        \psi[german \sq city, Set \rc Set] &= \lambda n{:} Name \abstr and \app (named \app n) \app (and \app inGermany \app city) \\
     \end{align*}

     Here we assign two different categories with their respective semantics
     to the phrase \code{german city} - one which can be seen in the query
     \code{"fountains in a german city"} and another in the query
     \code{"the german city Frankfurt"}.
}

Also, most examples from here on will assume that grammars have a starting
category $Set$, to be as close as possible to the implementation in
\cref{sec:minipass}.
\end{document}
