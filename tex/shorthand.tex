\documentclass[main.tex]{subfiles}
\begin{document}
\subsubsection{Parsing phrases}
\subsubsection{Shorthand for specifying grammars}
\label{shorthand}
When manually defining grammars, it is tedious to write definitions
for $f$ and $\psi$ using standard mathematical notation.

Thus, we will adopt the syntax on the left to mean the definitions on the right:
\begin{center}
    \begin{minipage}{.4\textwidth}
        \gramshort{
            \gramrow{\alpha_1}{X_{11}}{M_{11}}
            \gramrow{}{X_{12}}{M_{12}}
            &  & ... & \\
            \gramrow{}{X_1{m_1}}{M_1{m_1}}
            \gramrow{\alpha_2}{X_{21}}{M_{21}}
            \gramrow{}{X_{22}}{M_{22}}
            &  & ... & \\
            \gramrow{}{X_2{m_2}}{M_2{m_2}}
            &  ... & ... & \\
            \gramrow{\alpha_n}{X_{n1}}{M_{n1}}
            \gramrow{}{X_{n2}}{M_{n2}}
            &  & ... & \\
            \gramrow{}{X_n{m_n}}{M_n{m_n}}
        }
    \end{minipage}
    \hspace{.15\textwidth}
    \begin{minipage}{.4\textwidth}
        \begin{align*}
            f(\alpha_1) &= \{ X_{11}, X_{12}, ..., X_{1m_1} \} \\
            f(\alpha_2) &= \{ X_{21}, X_{22}, ..., X_{2m_2} \} \\
            ...& \\
            f(\alpha_n) &= \{ X_{n1}, X_{n2}, ..., X_{nm_n} \} \\
            \\
            \psi(\alpha_1, X_{11}) &= M_{11} \\
            \psi(\alpha_1, X_{12}) &= M_{12} \\
            \psi(\alpha_1, X_{1m_1}) &= M_{1m_1} \\
            \psi(\alpha_2, X_{21}) &= M_{21} \\
            \psi(\alpha_2, X_{22}) &= M_{22} \\
            \psi(\alpha_2, X_{2m_2}) &= M_{2m_2} \\
            ...& \\
            \psi(\alpha_n, X_{n1}) &= M_{n1} \\
            \psi(\alpha_n, X_{n2}) &= M_{n2} \\
            \psi(\alpha_n, X_{2m_n}) &= M_{nm_n} \\
        \end{align*}
    \end{minipage}
\end{center}
\example{
    This meta-example shows a couple of rules to illustrate the syntax:
    \gramshort{
        \gramrow{near}{ Set \lc Set \rc Set }{ \lambda from: Set, things: Set \abstr }
            \gramrow{}{}{ and \app things \app (within \app nearDistance \app from) }
        \gramrow{german \sq city}{ Set }{ and \app inGermany \app city }
        \gramrow{}{ Set \rc Name }{ \lambda n{:} Name \abstr and \app (named \app n) \app (and \app inGermany \app city) }
    }
    The grammar it specifies is the following:
    \begin{align*}
        f[german \sq city] &= \{ Set, Set \rc Name \} \\
        f(near) &= \{ Set \lc Set \rc Set \} \\
        \\
        \psi(near, Set \lc Set \rc Set) &= \lambda from: Set, things: Set \abstr \\
        & \phantom{=} \qquad and \app things \app (within \app nearDistance \app from) \\
        \psi[german \sq city, Set] &= and \app inGermany \app city \\
        \psi[german \sq city, Set \rc Set] &= \lambda n{:} Name \abstr and \app (named \app n) \app (and \app inGermany \app city) \\
     \end{align*}
}

Definitions for phrases such as \code{german cities} in the above example can
either be swept under the rug for the sake of example, or be assumed to use the
technique discussed in \cref{hack:phrases}.
\end{document}
