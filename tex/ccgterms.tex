\documentclass[main.tex]{subfiles}
\begin{document}
\subsubsection{Extracting $\lambda$-terms from CCG derivations}

A great strength of CCG comes from the ability to directly construct $\lambda$-terms
from CCG derivations. This relies on the assumption that the target
language has a compositional nature, namely that the semantics of any construction
can be determined by composing the semantics of its constituents.

So, we shall to provide the semantics for the basic blocks (in our case
the elements of $\Sigma$) in the form of $\lambda$-terms, and then define rules
for composing those terms in accordance with a CCG derivation tree.

\begin{defn}
    For a set of atomic categories $N$ typed in $\Theta$, we can extend the
    definition of the typing function $\typeoff$ over their categorial closure
    $\typeoff : \cclos{N} \rightarrow \tclos{\Theta}$ in the following way:
    $\typeof{\lp X \mc Y \rp} \defeq  \typeof{Y} \tot \typeof{X}$.

    This gives meaning to categories in the context of $\lambda$-calculus:
    $\lp X \lc Y \rp$ refers to a function that returns a value of type
    $X$ and takes an argument of type $Y$ \emph{on the right side}, while
    $\lp X \lc Y \rp$ refers to a function that returns a value of type
    $X$ and takes an argument of type $Y$ \emph{on the left side}.
\end{defn}

Now that we have assigned types to categories, we can attach $\lambda$-terms
to each terminal in addition to the categories we have attached via the function
$f$. This will give us $\lambda$-terms in the leaves of each derivation tree,
which can be composed naturally into $\lambda$-terms for all subtrees including
the whole tree, giving us the semantic information for this parse.

To do this, we will use a function that maps each terminal/category pair
to a $\lambda$-term. In order for the terms constructed on derivation trees
to be correct and consistent, we will also want the terms of each node to be
of the same type as its category.

\begin{defn}
    Let $\Theta$ be a set of type symbols,
    $ G = \langle \Sigma, N, S, f, n \rangle $ be a CCG, $N$ be
    a set of atomic categories typed in $\Theta$ and $C$ be a set
    of constants typed in $\Theta$.
    A function
    \[ \psi : \{ (a, X) \mid a \in \Sigma, X \in f(a) \} \rightarrow 2^{\plambda{\Theta}{C}} \] is called a
    \emph{semantic function} for G.
\end{defn}

\greenbox{
    While here we differentiate between \emph{categories} and their \emph{types},
    it is often sufficient for types and categories to be the same set:
    when that is the case, the typing function is simply identity.

    For simplicity, in most examples from here onward, the types of atomic
    categories will be the atomic categories themselves.
}

Now that we have our semantic function that can yield the possible terms
for each \emph{leaf} in a derivation tree, we will build the semantic function
for the entire tree:

\begin{defn}\label{def:semfn}
    Let $G$ be a CCG, $\psi$ be a semantic function for $G$ and $T$ be a
    derivation tree for $G$.

    We will inductively define $sem_{\psi}(T) \subset \plambda{\Theta}{C}$ with
    respect to $T$:
    \begin{enumerate}
        \item \label{sem:case:simple} If $T$ is
            \begin{center}
                \begin{tikzpicture}[node distance=1mm,sibling distance=1cm]
                    \Tree[
                        .\node(outtop){$X$};
                            \edge[very thick];
                            [ .\node(in){$a$};
                            ]
                    ]
                    \node[boxcolb,fit=(outtop)(in)](outbox){};
                    \node[right=of outbox,text=\boxtextb](outlabel){$T$};
                \end{tikzpicture}
            \end{center}
            and $X \in f(a)$, then
            \[ sem_{\psi}(T) = \psi(a, X). \]
        \item \label{sem:case:right} If $T$ is
            \begin{center}
                \begin{tikzpicture}[node distance=1mm,sibling distance=1cm]
                    \Tree[
                        .\node(outtop){$X \mci{1} Z_1 \mci{2} Z_2 ... \mci{m} Z_m$};
                            \edge[very thick];
                            [ .\node(intopl){$X \rc Y$};
                                \edge[roof];
                                [ .\node(inbotl){$\alpha$}; ]
                            ]
                            [ .\node(intopr){$Y \mci{1} Z_1 \mci{2} Z_2 ... \mci{m} Z_m$};
                                \edge[roof];
                                [ .\node(inbotr){$\beta$}; ]
                            ]
                    ]
                    \node[boxcola,fit=(intopl)(inbotl)](inboxl){};
                    \node[left=of inboxl,text=\boxtexta](inlabell){$T_1$};

                    \node[boxcola,fit=(intopr)(inbotr)](inboxr){};
                    \node[right=of inboxr,text=\boxtexta](inlabelr){$T_2$};

                    \node[boxcolb,fit=(outtop)(inbotl)(inboxl)(inboxr)(inlabell)(inlabelr)](outbox){};
                    \node[right=of outbox,text=\boxtextb](outlabel){$T$};
                \end{tikzpicture}
            \end{center}
            where $root(T_1) = X \rc Y$ and $root(T_2) = Y \mci{1} Z_1 \mci{2} Z_2 ... \mci{m} Z_m$ then
            \[
                \begin{split}
                    sem_{\psi}(T) =& \{
                        \lambda z_m {:} \typeof{Z_m}, z_{m - 1} {:} \typeof{Z_{m - 1}}, ..., z_1 {:} \typeof{Z_1} \abstr
                        M ( N z_m z_{m - 1} ... z_1 ) \\
                        & \quad \mid M \in sem_{\psi}(T_1), N \in sem_{\psi}(T_2) \} \\
                \end{split}
            \]
            for freshly-picked
            $z_1 ... z_m \in \lvars \setminus \big(\fv{sem_{\psi}(T_1)} \cup \fv{sem_{\psi}(T_2)}\big)$.
        \item \label{sem:case:left} If $T$ is
            \begin{center}
                \begin{tikzpicture}[node distance=1mm,sibling distance=1cm]
                    \Tree[
                        .\node(outtop){$X \mci{1} Z_1 \mci{2} Z_2 ... \mci{m} Z_m$};
                            [ .\node(intopl){$Y \mci{1} Z_1 \mci{2} Z_2 ... \mci{m} Z_m$};
                                \edge[roof];
                                [ .\node(inbotl){$\alpha$}; ]
                            ]
                            \edge[very thick];
                            [ .\node(intopr){$X \lc Y$};
                                \edge[roof];
                                [ .\node(inbotr){$\beta$}; ]
                            ]
                    ]
                    \node[boxcola,fit=(intopl)(inbotl)](inboxl){};
                    \node[left=of inboxl,text=\boxtexta](inlabell){$T_1$};

                    \node[boxcola,fit=(intopr)(inbotr)](inboxr){};
                    \node[right=of inboxr,text=\boxtexta](inlabelr){$T_2$};

                    \node[boxcolb,fit=(outtop)(inbotl)(inboxl)(inboxr)(inlabell)(inlabelr)](outbox){};
                    \node[right=of outbox,text=\boxtextb](outlabel){$T$};
                \end{tikzpicture}
            \end{center}
            where $root(T_1) = Y \mci{1} Z_1 \mci{2} Z_2 ... \mci{m} Z_m$ and $root(T_2) = X \lc Y$ then
            \[
                \begin{split}
                    sem_{\psi}(T) =& \{
                        \lambda z_m {:} \typeof{Z_m}, z_{m - 1} {:} \typeof{Z_{m - 1}}, ..., z_1 {:} \typeof{Z_1} \abstr
                        N ( M z_m z_{m - 1} ... z_1 ) \\
                        & \quad \mid M \in sem_{\psi}(T_1), N \in sem_{\psi}(T_2) \} \\
                \end{split}
            \]
            for freshly-picked
            $z_1 ... z_m \in \lvars \setminus \big(\fv{sem_{\psi}(T_1)} \cup \fv{sem_{\psi}(T_2)}\big).$
    \end{enumerate}
\end{defn}

Observe that since $\psi$ returns valid $\lambda$-terms and everything
returned by $sem_{\psi}$ is constructed via basic $\lambda$-calculus operations,
$sem_{\psi}$ returns only valid $\lambda$-terms.

While it is easier to only compose closed terms, there are cases when free
variables are useful. We will not restrict ourselves to closed terms and will
therefore take some extra care to make this construction support free variables.

One way to do this is to have a global context for all terms that are given
by the semantic function: this way we have a universal way for finding the
type of each subterm, even if it contains free variables. Then, consistency
between the terms returned by $\psi$ and their categories can be defined
cleanly:

\begin{defn}
    A semantic function $\psi$ is called \emph{consistent in} a context
    $\Gamma$ when
    \[
        \forall a, X: M \in \psi(a, X) \implies \Gamma \vdash M : \typeof{X}.
    \]
\end{defn}

Another observation is that the construction for $sem_{\psi}$ includes no
additional free variables other than the ones already present in the terms
returned by $\psi$.

\begin{prop}
    \label{ccg:typeconsistent}
    If $ G = \langle \Sigma, N, S, f, n \rangle $ is a CCG and $\psi$ is a
    semantic function for $G$ that is consistent in $\Gamma$, then
    for every valid derivation tree $T$:
    \[
        M \in sem_{\psi}(T) \implies \Gamma \vdash M : \typeof{root(T)}
    \]
\end{prop}
\begin{proof}
    This can be proven by induction over the definition of $sem_{\psi}$.

    Let $L \in sem_{\psi}(T)$ be a term produced by $sem_{\psi}$ for an
    arbitrary derivation tree $T$. It has been constructed in one of the
    following ways:
    \begin{itemize}
        \item $L$ was produced by case \ref{sem:case:simple}. Then
            $L \in \psi(a, X)$ for some $X \in f(a)$. $\psi$
            is consistent in $\Gamma$, so $\Gamma \vdash L: \typeof{X}$. Also, since
            $root(T) = X$, $\typeof{root(T)} = \typeof{X}$. Thus,
            $\Gamma \vdash L: \typeof{root(T)}$
        \item $L$ was produced by case \ref{sem:case:right}. Then,
            inductively we have:
            \begin{equation}
                \label{eq:semr:lt}
                \Gamma \vdash M : \typeof{X \rc Y} = \typeof{Y} \tot \typeof{X}
            \end{equation}
            and
            \begin{equation}
            \label{eq:semr:rt}
                \Gamma \vdash N : \typeof{Y \mci{1} Z_1 \mci{2} Z_2 ... \mci{m} Z_m}
                = \typeof{Z_m} \tot ... \tot \typeof{Z_1} \tot \typeof{Y}
            \end{equation}
            Aside from that,
            \begin{equation}
                L = \lambda z_m {:} \typeof{Z_m}, z_{m - 1} {:} \typeof{Z_{m - 1}}, ..., z_1 {:} \typeof{Z_1} \abstr
                \underbrace{M ( N z_m z_{m - 1} ... z_1 )}_{K}
            \end{equation}
            Let $\Gamma' = \Gamma \circ z_m : \typeof{Z_m} \circ ... \circ
                           z_2 : \typeof{Z_2} \circ z_1 : \typeof{Z_1}$.
            Since $z_1 ... z_m \not\in \fv{M} \cup \fv{N}$ because we pick them
            as fresh variables, \cref{eq:semr:lt} holds for $\Gamma'$ in the
            same way it does for $\Gamma$.

            So, $\Gamma' \vdash N z_m z_{m - 1} ... z_1 : \typeof{Y}$ and then
            $\Gamma' \vdash \underbrace{M ( N z_m z_{m - 1} ... z_1 )}_{K} : \typeof{X}$.

            By the definition of $\vdash$ for the case of abstraction,
            \begin{equation}
                \Gamma \vdash \underbrace{(\lambda z_m {:} \typeof{Z_m}, z_{m - 1} {:} \typeof{Z_{m - 1}}, ..., z_1 {:} \typeof{Z_1} \abstr K)}_{L} :
                \typeof{Z_m} \tot ... \tot \typeof{Z_1} \tot \typeof{X}
            \end{equation}
            But $\typeof{root(T)} = \typeof{X \mci{1} Z_1 \mci{2} Z_2 ... \mci{m} Z_m}
            = \typeof{Z_m} \tot ... \tot \typeof{Z_1} \tot \typeof{X}$,
            therefore $\Gamma \vdash L : \typeof{root(T)}$ which is exactly what
            we sought to prove.
        \item $L$ was produced by case \ref{sem:case:left}. Since it is completely
            analogous to case \ref{sem:case:right}, the proof will be omitted.
    \end{itemize}
\end{proof}

Given proposition \autoref{ccg:typeconsistent}, we know that all terms generated
from CCG derivation trees will be consistent.

\end{document}
