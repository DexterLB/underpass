\documentclass[main.tex]{subfiles}
\begin{document}

\subsection{\hsout{Hacks} Extensions of the CCG formalism}
\label{sec:hacks}

The CCG formalism described in \cref{sec:ccg} is too weak to represent most
phenomena in natural languages \cite{steedman}. Thus, several extensions have
been developed to deal with them.

\subsubsection{Rule restrictions or modalities}\label{hack:modal}
The only restriction on rules within the pure CCG formalism is the limit to
the number of arguments ($n$). However, additional rule restrictions are often
added in practice, which makes the grammar more expressive \cite{restrictions}.

The simplest method to implement this is to allow arbitrary rule restrictions,
essentially creating a new formalism for each new target language.
Another method is the so called \emph{slash modalities}, which have been shown
\cite{modal}
to be equivalent to rule restrictions in terms of expressive power, and will
be used here.

\begin{defn}
    The elements of the set
    $\mathcal{M} = \{ \modstar, \modr, \modx, \moddot \}$ are called
    \emph{slash modalities}.

    Slash modalities make up the following lattice:

    \begin{center}
        \begin{tikzpicture}
            \node[main node] (1) {$\modstar$};
            \node[main node] (2) [below left = 1cm and 1.5cm of 1]  {$\modr$};
            \node[main node] (3) [below right = 1cm and 1.5cm of 1] {$\modx$};
            \node[main node] (4) [below = 2cm of 1] {$\moddot$};

            \path[draw,thick]
            (1) edge node {} (2)
            (1) edge node {} (3)
            (2) edge node {} (4)
            (3) edge node {} (4);
        \end{tikzpicture}
    \end{center}
\end{defn}

For clarity, slash indices will be denoted with latin letters ($\mci{i}$
denotes the $i$'th respective slash, including its direction and modality)
while slash modalities will be denoted by Greek letters
($\lci{\mu}$ and  $\rci{\mu}$ denote the respective
slash with modality $\mu$).

\begin{defn}
    We extend the definition of categorial closure to use modalities:
    \begin{enumerate}
        \item \label{cmod:atomic} $A \in \tau \implies A \in \cclos{\tau}$
        \item \label{cmod:right}  $X, Y \in \cclos{\tau}, \mu \in \mathcal{M} \implies \lp X \rci{\mu} Y \rp \in \cclos{\tau}$
        \item \label{cmod:left}   $X, Y \in \cclos{\tau}, \mu \in \mathcal{M} \implies \lp X \lci{\mu} Y \rp \in \cclos{\tau}$
    \end{enumerate}
\end{defn}

Now, we restrict derivations according to modalities (in all cases, $m \leq n$):
\begin{itemize}
    \item ``Application''
        \[ \lb X \rb \gderiv \lb X \rci{\mu} Y \rb \lb Y \rb \]
        \[ \lb X \rb \gderiv \lb Y \rb \lb X \lci{\mu} Y \rb \]
        where $\mu \less \modstar$ (effectively for any $\mu$).
    \item ``Harmonic composition''
        \[ \lb X \rci{\eta} Z_1 \mci{2} Z_2 ... \mci{m} Z_m \rb \gderiv \lb X \rci{\mu} Y \rb \lb Y \rci{\nu} Z_1 \mci{2} Z_2 ... \mci{m} Z_m \rb \]
        \[ \lb X \lci{\eta} Z_1 \mci{2} Z_2 ... \mci{m} Z_m \rb \gderiv \lb Y \lci{\mu} Z_1 \mci{2} Z_2 ... \mci{m} Z_m \rb \lb X \lci{\nu} Y \rb \]
        where $\mu \less \modr, \nu \less \modr, \eta = \mu \meet \nu$.
    \item ``Crossing composition''
        \[ \lb X \lci{\eta} Z_1 \mci{2} Z_2 ... \mci{m} Z_m \rb \gderiv \lb X \rci{\mu} Y \rb \lb Y \lci{\nu} Z_1 \mci{2} Z_2 ... \mci{m} Z_m \rb \]
        \[ \lb X \rci{\eta} Z_1 \mci{2} Z_2 ... \mci{m} Z_m \rb \gderiv \lb Y \rci{\mu} Z_1 \mci{2} Z_2 ... \mci{m} Z_m \rb \lb X \lci{\nu} Y \rb \]
        where $\mu \less \modx, \nu \less \modx, \eta = \mu \meet \nu$.
\end{itemize}

In the above templates, modalities on the right-hand side stand for any
modality that is less-than or equal (according to the modality lattice)
to the one specified. The modalities on the left-hand side are produced by
unifying (lowest common ancestor) the modalities on the right.

\begin{example}
    For example, here are a few valid derivations:
    \[ \lb X \rb \gderiv \lb X \rci{\modr} Y \rb \lb Y \rb \]
    \[ \lb X \rb \gderiv \lb X \rci{\modx} Y \rb \lb Y \rb \]
    \[ \lb X \rb \gderiv \lb X \rci{\moddot} Y \rb \lb Y \rb \]
    \[ \lb X \rci{\moddot} Z_1 \mci{2} Z_2 ... \mci{m} Z_m \rb \gderiv \lb X \rci{\modr} Y \rb \lb Y \rci{\moddot} Z_1 \mci{2} Z_2 ... \mci{m} Z_m \rb \]
    \[ \lb X \rci{\moddot} Z_1 \mci{2} Z_2 ... \mci{m} Z_m \rb \gderiv \lb Y \rci{\modx} Z_1 \mci{2} Z_2 ... \mci{m} Z_m \rb \lb X \lci{\moddot} Y \rb \]
\end{example}

This way, one can restrict the ways in which a category would
combine by choosing appropriate slash modalities.

When using slash modalities, the bare slashes ($\lc$ and $\rc$) will actually
mean $\lci{\moddot}$ and $\rci{\moddot}$, being the most generic (they
can combine with any other type of slash).

\begin{example}
    Consider the following simple grammar:
    \gramshort{
        \gramrow{small}{ GSet \rc GSet }{}
        \gramrow{cities}{ GSet }{}
        \gramrow{in}{ GSet \lc GSet \rc GSet }{}
        \gramrow{Russia}{ GSet }{}
    }

    It generates the following query, which looks well:
    \centree{.{$GSet$}
        [ .{$GSet$}
            [ .{$GSet \rc GSet$} [ .{$small$} ] ]
            [ .{$GSet$} [ .{$cities$} ] ]
        ]
        [ .{$GSet \lc GSet$}
            [ .{$GSet \lc GSet \rc GSet$} [ .{$in$} ] ]
            [ .{$GSet$} [ .{$Russia$} ] ]
        ]
    }

    However, it also generates this query, which is undesirable:
    \centree{.{$GSet$}
        [ .{$GSet$} [ .{$cities$} ] ]
        [ .{$GSet \lc GSet$}
            [ .{$GSet \rc GSet$} [ .{$small$} ] ]
            [ .{$GSet \lc GSet$}
                [ .{$GSet \lc GSet \rc GSet$} [ .{$in$} ] ]
                [ .{$GSet$} [ .{$Russia$} ] ]
            ]
        ]
    }

    It appears that adjectives such as $small$ should not be allowed to combine
    in this way. This can be remedied by setting $small \catdefsep GSet \rci{\modr} GSet$
    in order to disallow crossing composition.
\end{example}

\subsubsection{Categorial variables}
\label{hack:catvars}
This is a very powerful extension that adds simple unification based on variables
to categorial derivation. To define it formally, let $\cvars = \{ \kappa', \kappa'', \kappa''' ...\}$
be a countable set of symbols that we will call \emph{categorial variables}.

\begin{defn}
    We extend the definition of categorial closure to include variables:
    \begin{enumerate}
        \item \label{cvar:atomic} $A \in \tau \implies A \in \cclos{\tau}$
        \item \label{cvar:right}  $X, Y \in \cclos{\tau} \implies \lp X \rc Y \rp \in \cclos{\tau}$
        \item \label{cvar:left}   $X, Y \in \cclos{\tau} \implies \lp X \lc Y \rp \in \cclos{\tau}$
        \item \label{cvar:var}    $\alpha \in \cvars \implies \alpha \in \cclos{\tau}$
    \end{enumerate}
\end{defn}

For convenience, we will use the same notation as the function returning the free variables of $\lambda$-terms, for categories:
\begin{defn}
    Let $\fvv : \cclos{\tau} \fun 2^{\cvars}$ be defined as:
    \[
        \fv{Z} =
        \begin{cases*}
            \emptyset ,& $Z \in \tau$ \\
            \fv{X} \cup \fv{Y} ,& $Z = \lp X \mc Y \rp$ \\
            \{ \alpha \} ,& $Z = \alpha \in \cvars$ \\
        \end{cases*}
    \]
\end{defn}

We now need to define \emph{variable substitution} over categories in order
to be able to easily manipulate variables.
\begin{defn}
    Variable substitution over categories is defined as
    \[
        \subst{Z}{\alpha}{W} =
        \begin{cases*}
            Z ,& $Z \in \tau$ \\
            \lp \subst{X}{\alpha}{W} \mc \subst{Y}{\alpha}{W} \rp ,& $Z = \lp X \mc Y \rp$ \\
            \beta ,& $Z = \beta \in \cvars, \beta \neq \alpha$ \\
            W ,& $Z = \alpha$ \\
        \end{cases*}
    \]
    for $Z \in \cclos{\tau}, W \in \cclos{\tau}, \alpha \in \cvars$
\end{defn}

\begin{convention}
    If we have
    \[ X' = \subst{\subst{\subst{X}{\alpha_1}{Y_1}}{\alpha_2}{Y_2}...}{\alpha_n}{Y_n}, \]
    then we will use
    \[ \xi = \subst{\subst{\subst{}{\alpha_1}{Y_1}}{\alpha_2}{Y_2}...}{\alpha_n}{Y_n} \]
    to denote that ``$\xi$ is the sequence of substitutions
    $\subst{\subst{\subst{}{\alpha_1}{Y_1}}{\alpha_2}{Y_2}...}{\alpha_n}{Y_n}$''.

    The application of $\xi$ on $X$ will be denoted as
    \[ X' = \xi(X). \]
\end{convention}

Having extended the notion of categorial closure to include categorial variables,
we also add a family of derivation rules to deal with variables, namely:
\centree{.{$\subst{X}{\alpha}{Y}$} \edge[very thick]; {$X$} }
for any $X \in \cclos{\tau}, Y \in \cclos{\tau}, \alpha \in \cvars$.

To define semantics for derivations with categorial variables, we need to extend
our $\lambda$-calculus to include type-variables and type-variable substitution.
For a more detailed explanation, see \cref{lambda:typevars}.
Also, $\typeoff$ for categories must be extended to injectively map categorial
variables ($\cvars$) into type variables ($\tvars$) for variable substitution
to work (and for category types that contain variables to match the types of
their assigned terms).

Then, the semantics for the following derivation tree produced by variable
substitution can be defined:
\begin{center}
    \begin{tikzpicture}[node distance=1mm]
        \Tree[
            .\node(outtop){$\subst{X}{a}{Y}$};
                \edge[very thick];
                [ .\node(intop){$X$};
                    \edge[roof];
                    [ .\node(inbot){$\alpha$}; ]
                ]
        ]
        \node[boxcola,fit=(intop)(inbot)](inbox){};
        \node[right=of inbox,text=\boxtexta](inlabel){$T'$};

        \node[boxcolb,fit=(outtop)(inbot)(inbox)(inlabel)](outbox){};
        \node[right=of outbox,text=\boxtextb](outlabel){$T$};
    \end{tikzpicture}
\end{center}

\begin{align*}
    sem_{\psi}(T) &= \{ \subst{M}{\typeof{a}}{Y} \mid M \in sem_{\psi}(T') \} \\
    root(T) &= \subst{X}{a}{Y} \\
    crown(T) &= crown(T') = \alpha \\
\end{align*}

This essentially allows us to generate any category and substitute it in place
of variables at any time during derivation.

When parsing, we can observe that after processing all leaf nodes,
there is no way to generate new variables while building the parse tree
bottom-up. Thus, it is sufficient to eliminate variables by unification:

\begin{defn}\label{def:varunify}
Two categories $X$ and $Y$ with categorial variables \emph{unify} whenever we can
find two sequences of substitutions $(a_1, X_1), (a_2, X_2), ..., (a_n, X_n)$
and $(b_1, Y_1), (b_2, Y_2), ..., (b_n, Y_n)$ such that
\[ \subst{\subst{\subst{X}{a_1}{X_1}}{a_2}{X_2}}{a_n}{X_n}
 = \subst{\subst{\subst{Y}{b_1}{Y_1}}{b_2}{Y_2}}{b_n}{Y_n} \]
\end{defn}

In order to implement parsing of CCGs with variables, we will modify the categorial
CYK algorithm by replacing the rules (\cref{cyk:rules}) by the following:

\begin{enumerate}
    \item If $X \in f(w_i)$, then $(X, i, i) \in P$.
    \item If $(X \rc Y', i, p) \in P, (Y'' \mci{1} Z_1 \mci{2} Z_2 ... \mci{m} Z_m, p + 1, j) \in P$
        and $Y'$ and $Y''$ unify with substitutions
        $ {Y'} \underbrace{\subst{\subst{\subst{}{a_1}{{Y'}_1}}{a_2}{{Y'}_2}}{a_n}{{Y'}_n}}_{\xi'}
        = {Y''}\underbrace{\subst{\subst{\subst{}{b_1}{{Y''}_1}}{b_2}{{Y''}_2}}{b_n}{{Y''}_n}}_{\xi''}$, \\
        then $(\xi'(X) \mci{1} \xi''(Z_1) \mci{2} \xi''(Z_2) ... \mci{m} \xi''(Z_m)), i, j) \in P$.
        \fixednote{Тук не трябва ли да е $(\xi'(X) \mci{1} \xi''(Z_1) \mci{2} \xi''(Z_2) ... \mci{m} \xi''(Z_m), i, j)\in P$ заради лявата асоциативност?}
    \item If $(X \lc Y', p + 1, j) \in P, (Y'' \mci{1} Z_1 \mci{2} Z_2 ... \mci{m} Z_m, i, p) \in P$
        and $Y'$ and $Y''$ unify with substitutions
        $ {Y'} \underbrace{\subst{\subst{\subst{}{a_1}{{Y'}_1}}{a_2}{{Y'}_2}}{a_n}{{Y'}_n}}_{\xi'}
        = {Y''}\underbrace{\subst{\subst{\subst{}{b_1}{{Y''}_1}}{b_2}{{Y''}_2}}{b_n}{{Y''}_n}}_{\xi''}$, \\
        then $(\xi'(X) \mci{1} \xi''(Z_1) \mci{2} \xi''(Z_2) ... \mci{m} \xi''(Z_m)), i, j) \in P$.
\end{enumerate}

\subsubsection{Coordination}
\emph{Coordination} refers to a technique that allows conjunctions (such as
$and$, $or$ etc.) to combine identical categories on both sides.

This can be emulated completely by using categorial variables and setting
$and \catdefsep \alpha \lc \alpha \rc \alpha $\footnote{
    In practice, one would need to set
    $and \catdefsep X \lc X \rc X, \lp X \mc \alpha \rp \lc \lp X \mc \alpha \rp \rc \lp X \mc \alpha \rp, ...$
    in order to properly define semantics.
} and similar.

\subsubsection{Type-raising}
This extension adds a family of unary rules for constructing derivations \cite[sec.~5.3.1]{nts},
parametrised for any slash modality $i$:
\begin{center}
    \tree{.{$\alpha \rci{i} \lp \alpha \lci{i} X \rp$} \edge[very thick]; {$X$} }
        ( Forward type-raising )
    \tree{.{$\alpha \lci{i} \lp \alpha \rci{i} X \rp$} \edge[very thick]; {$X$} }
        ( Backward type-raising )
\end{center}

The semantics are defined as follows:
\begin{center}
    \begin{tikzpicture}[node distance=1mm]
        \Tree[
            .\node(outtop){$\alpha \rci{i} \lp \alpha \lci{i} X \rp$};
                \edge[very thick];
                [ .\node(intop){$X$};
                    \edge[roof];
                    [ .\node(inbot){$\alpha$}; ]
                ]
        ]
        \node[boxcola,fit=(intop)(inbot)](inbox){};
        \node[right=of inbox,text=\boxtexta](inlabel){$T'$};

        \node[boxcolb,fit=(outtop)(inbot)(inbox)(inlabel)](outbox){};
        \node[right=of outbox,text=\boxtextb](outlabel){$T$};
    \end{tikzpicture}
    \hspace{2cm}
    \begin{tikzpicture}[node distance=1mm]
        \Tree[
            .\node(outtop){$\alpha \lci{i} \lp \alpha \rci{i} X \rp$};
                \edge[very thick];
                [ .\node(intop){$X$};
                    \edge[roof];
                    [ .\node(inbot){$\alpha$}; ]
                ]
        ]
        \node[boxcola,fit=(intop)(inbot)](inbox){};
        \node[right=of inbox,text=\boxtexta](inlabel){$T'$};

        \node[boxcolb,fit=(outtop)(inbot)(inbox)(inlabel)](outbox){};
        \node[right=of outbox,text=\boxtextb](outlabel){$T$};
    \end{tikzpicture}
\end{center}

\[ sem_{\psi}(T) = \{ \lambda f {:} ( X \tot \alpha ) \abstr f M \mid M \in sem_{\psi}(T') \} \]

Since considering type-raising at any level would make parsing much slower,
it is usually only considered on categories that appear in leaves.

\begin{example}
    Consider the following simple grammar:
    \gramshort{
        \gramrow{cities}{ GSet }{}
        \gramrow{villages}{ GSet }{}
        \gramrow{Finland}{ GSet }{}
        \gramrow{and}{ \alpha \lc \alpha \rc \alpha }{}
        \gramrow{in}{ GSet \lc GSet \rc GSet }{}
        \gramrow{near}{ GSet \lc GSet \rc GSet }{}
    }
    It permits derivations as such:
    \centree{
        .{$GSet$}
            \edge[very thick];
            [ .{$GSet$}
                [ .{$GSet$}
                    [ .{$cities$} ]
                ]
                [ .{$GSet \lc GSet$}
                    [ .{$GSet \lc GSet \rc GSet$}
                        [ .{$\alpha \lc \alpha \rc \alpha$} [ .{$and$} ] ]
                    ]
                    [ .{$GSet$}
                        [ .{$villages$} ]
                    ]
                ]
            ]
            [ .{$GSet \lc GSet$}
                [ .{$GSet \lc GSet \rc GSet$}
                    [ .{$in$} ]
                ]
                [ .{$GSet$}
                    [ .{$Finland$} ]
                ]
            ]
    }

    If we would like to be able to accept queries like ``cities in and villages
    near Finland'', we could permit type-raising:

    \autoscaledtree{
        .{$GSet$}
        [ .{$GSet \rc GSet$}
            [ .{$ GSet \rc GSet$}
                [ .{$GSet \rc \lp GSet \lc GSet \rp$}
                    [ .{$\alpha \rc \lp \alpha \lc GSet \rp$} [ .{$GSet$} [ .{$cities$} ] ] ]
                ]
                [ .{$GSet \lc GSet \rc GSet$} [ .{$in$} ]
                ]
            ]
            [ .{$ \lp GSet \rc GSet \rp \lc \lp GSet \rc GSet \rp$}
                [ .{$\lp GSet \rc GSet \rp \lc \lp GSet \rc GSet \rp \rc \lp GSet \rc GSet \rp$}
                    [ .{$\alpha \lc \alpha \rc \alpha$} [ .{$and$} ] ]
                ]
                    [ .{$GSet \rc GSet$}
                    [ .{$GSet \rc \lp GSet \lc GSet \rp$}
                        [ .{$\alpha \rc \lp \alpha \lc GSet \rp$} [ .{$GSet$} [ .{$villages$} ] ] ]
                    ]
                    [ .{$GSet \lc GSet \rc GSet$} [ .{$near$} ]
                    ]
                ]
            ]
        ]
        [ .{$ GSet$} [ .{$Finland$} ] ]
    }
\end{example}

\subsubsection{Equification of categories}
\label{hack:equification}
To reduce the size and of grammars and make them less complicated,
it is convenient to regard
not only identical categories as equal, but rather use an \emph{equification
relation} that can judge whether two categories must be regarded as equal.

One method is to explicitly state the equification rules in the grammar.
It is also useful to make categories complex objects and infer the relation
based on their structure, as will be done in the next two sections.

\begin{defn}
    A relation $U \subseteq N \times N$ is called an
    \emph{equification relation} for the set $N$ if it is transitive
    and reflexive.
\end{defn}

To make use of this relation for parsing, we must extend it over complex categories.
\fixednote{тук не се разбира какво имаш предвид}
One way to define such an extension is by uniformly propagating membership
into $U$ over complex categories:
\begin{defn}\label{def:rel:uniform}
    If $U \subseteq N \times N$ is an equification relation
    for the set of
    atomic categories $N$, its extension over complex categories
    $\hat{U} \subseteq \cclos{N} \times \cclos{N}$ is defined
    inductively as follows:\fixednote{за индуктивни дефиниции досега си използвал стила със клаузи, защо не го ползваш и тук?}
    \begin{itemize}
        \item $X \in N \land Y \in N \land U(X, Y) \implies \hat{U}(X, Y)$
        \item $X = (W' \mc Z') \land Y = (W'' \mc Z'') \land \hat{U}(W', W'')
        \land \hat{U}(Z', Z'') \implies \hat{U}(X, Y)$
    \end{itemize}
\end{defn}

\begin{property}
    If $U \subseteq N \times N$ is an equification relation
    for the set of
    $N$, its extension $\hat{U}$ is an equification relation for $\cclos{U}$
\end{property}
\begin{proof}
    The arguments for transitivity and reflexivity are both trivial.
\end{proof}

Another way to extend an equification relation is presented in \cref{hack:subtyping}.

Now, we will modify the derivation rules from \cref{def:ccg:rules} to incorporate
equification:

\begin{itemize}
    \item If $ a \in \Sigma, X \in f(a) $, then \[ \lb X \rb \gderiv a \]
    \item If $ X \rc Y' \in \cclos{N}, Y'' \mci{1} Z_1 \mci{2} Z_2 ... \mci{m} Z_m \in \cclos{N}, 0 \leq m \leq n $
        and $\hat{U}(Y', Y'')$, \\
        then \[ \lb X \mci{1} Z_1 \mci{2} Z_2 ... \mci{m} Z_m \rb \gderiv \lb X \rc Y \rb \lb Y \mci{1} Z_1 \mci{2} Z_2 ... \mci{m} Z_m \rb \]
    \item If $ X \lc Y' \in \cclos{N}, Y'' \mci{1} Z_1 \mci{2} Z_2 ... \mci{m} Z_m \in \cclos{N}, 0 \leq m \leq n $
        and $\hat{U}(Y', Y'')$, \\
        then \[ \lb X \mci{1} Z_1 \mci{2} Z_2 ... \mci{m} Z_m \rb \gderiv \lb Y \mci{1} Z_1 \mci{2} Z_2 ... \mci{m} Z_m \rb \lb X \lc Y \rb \]
\end{itemize}

To implement parsing, we can replace the rules in the categorial CYK algorithm
(\cref{cyk:rules}) by the following:

\begin{enumerate}
    \item If $X \in f(w_i)$, then $(X, i, i) \in P$
    \item If $(X \rc Y', i, p) \in P, (Y'' \mci{1} Z_1 \mci{2} Z_2 ... \mci{m} Z_m, p + 1, j) \in P$
        and $\hat{U}(Y', Y'')$, \\
        then $(X \mci{1} Z_1 \mci{2} Z_2 ... \mci{m} Z_m, i, j) \in P$
    \item If $(X \lc Y', p + 1, j) \in P, (Y'' \mci{1} Z_1 \mci{2} Z_2 ... \mci{m} Z_m, i, p) \in P$
        and $\hat{U}(Y', Y'')$, \\
        then $(X \mci{1} Z_1 \mci{2} Z_2 ... \mci{m} Z_m, i, j) \in P$
\end{enumerate}

Note that the equification relation may be asymmetric and take advantage of the fact
that its first argument is the tail of the primary category.

\subsubsection{Equification by category features}
Some models allow each atomic category to be extended with a set of \emph{features}
that refine its scope, and then define rules for unification of said features.

For example, we could assume a global set of features $Feat$ to be the
set of finite functions $f: [a-zA-Z\_]^* \fun [a-zA-Z\_]^*$
and instead of using $\cclos{\tau}$ directly, we use $\cclos{\tau \times Feat}$.
For brevity, categories like $(V, \{ (tense, past), (person, first) \})$
are written as $V[tense=past, person=first]$.

A common equification scheme is to check whether a union of the two feature functions,
is also a function --- this determines whether the categories actually equify.

This can be further extended by concepts such as \emph{feature variables}
(which allows the use of variables within feature expressions and
performs variable substitution during unification) and embedding regular expressions.

\subsubsection{Equification by subtyping}\label{hack:subtyping}
If the grammar will be used for generating typed $\lambda$-terms that use
a subtyping system like the one described in \cref{sec:lambda}, the subtyping
relation can be used as an equification relation.

If the desired semantics is to allow functions to only accept subtypes of
their argument, some extra care needs to be taken, namely:
\[ \lb X \mci{1} Z_1 \mci{2} Z_2 ... \mci{m} Z_m \rb \gderiv \lb X \rc Y_1 \rb \lb Y_2 \mci{1} Z_1 \mci{2} Z_2 ... \mci{m} Z_m \rb \]
\[ \lb X \mci{1} Z_1 \mci{2} Z_2 ... \mci{m} Z_m \rb \gderiv \lb Y_2 \mci{1} Z_1 \mci{2} Z_2 ... \mci{m} Z_m \rb \lb X \lc Y_1 \rb \]
if and only if\note{след като сменяш дефиницията на правилата за извод, може би трябва да смениш и дефиницията на $sem_\psi$, ако искаш свойство \cref{ccg:typeconsistent} да остане в сила?}
\begin{equation}\label{eq:typeless}
    \typeof{Y_2} \less \typeof{Y_1}
\end{equation}

If we use an equification relation $U(X, Y) \defiff \typeof{Y} \less \typeof{X}$,
it is not sufficient to use the uniform extension to complex categories
(\cref{def:rel:uniform}) because that would violate contravariance. Thus,
we will use the following extension:
\begin{defn}
    If $U \subseteq N \times N$ is an equification relation
    for the set of
    atomic categories $N$, its contravariant extension over complex categories
    $\overline{U} \subseteq \cclos{N} \times \cclos{N}$ is defined
    inductively as follows:\fixednote{също препоръчвам дефиниция с клаузи}
    \begin{itemize}
        \item $X \in N \land Y \in N \land U(X, Y) \implies \overline{U}(X, Y)$
        \item $X = (W' \mc Z') \land Y = (W'' \mc Z'') \land \overline{U}(W'', W')
        \land \overline{U}(Z', Z'')\implies \overline{U}(X, Y)$
    \end{itemize}
\end{defn}

It can be easily seen that using
$\overline{U}$ is equivalent
to the condition in \cref{eq:typeless}
(given that $U(X, Y) \defiff \typeof{Y} \less \typeof{X}$).
\fixednote{запетайка в началото на реда изглежда странно}
\begin{example}
    Imagine we want to be able to parse the following queries:
    \begin{itemize}
        \item \code{German \sq cities \sq less \sq than \sq 20km \sq from \sq Cologne}
        \item \code{German \sq cities \sq less \sq than \sq 20km \sq north \sq of \sq Cologne}
    \end{itemize}

    For the sake of the example, subtrees for phrases like $German \sq cities$
    will be squashed and constants like $germanCity$, $within$ etc will be
    used without implementations\footnote{
        This example uses minipass-like semantics for $\lambda$-terms to
        better illustrate the application. See \cref{sec:minipass}.
    }.

    The grammar uses the following subtype definitions:
    \begin{align*}
        DistCheck \quad\lass\quad& ((Dist \tot GSet \tot GSet) \tot GSet) \\
        NorthCheck \quad\lass\quad& ((Dist \tot GSet \tot GSet) \tot GSet) \\
    \end{align*}
    and looks like this:
    \gramshort{
        \gramrow{German \sq cities}{GSet}{germanCity}
        \gramrow{Cologne}{GSet}{cologne}
        \gramrow{20km}{Dist}{twentykm}
        \gramrow{less \sq than}{GSet \lc GSet \rc DistCheck}{
            \lambda b {:} DistCheck, s{:} GSet \abstr}
            \gramrow{}{}{ \quad b \app (
                \lambda d {:} Dist, c {:} GSet \abstr
            }
            \gramrow{}{}{ \quad \quad and \app s \app (within \app d \app c)}
            \gramrow{}{}{ \quad )}
        \gramrow{}{GSet \lc GSet \rc NorthCheck}{
            \lambda b {:} NorthCheck, s{:} GSet \abstr}
            \gramrow{}{}{ \quad b \app (
                \lambda d {:} Dist, c {:} GSet \abstr
            }
            \gramrow{}{}{ \quad \quad and \app s \app (withinNorth \app d \app c)}
            \gramrow{}{}{ \quad )}
        \gramrow{from}{DistCheck \lc Dist \rc GSet}{
            \lambda c {:} GSet, d {:} Dist \abstr DistCheck [}
            \gramrow{}{}{ \quad \lambda f {:} (Dist \tot GSet \tot GSet) \abstr f \app d \app c }
            \gramrow{}{}{]}
        \gramrow{north \sq of}{NorthCheck \lc Dist \rc GSet}{
            \lambda c {:} GSet, d {:} Dist \abstr NorthCheck [}
            \gramrow{}{}{ \quad \lambda f {:} (Dist \tot GSet \tot GSet) \abstr f \app d \app c }
            \gramrow{}{}{]}
    }

    Here is a derivation tree for one of the aforementioned queries:
    \centree{.{$GSet$}
        [ .{$GSet$} \edge[roof]; [ .{$German \sq cities$} ] ]
        [ .{$GSet \lc GSet$}
            [ .{$GSet \lc GSet \rc DistCheck$} \edge[roof]; [ .{$less \sq than$} ] ]
            [ .{$DistCheck$}
                [ .{$Dist$} [ .{$20km$} ] ]
                [ .{$DistCheck \lc Dist$}
                    [ .{$DistCheck \lc Dist \rc GSet$} [ .{$from$} ] ]
                    [ .{$GSet$} [ .{$Cologne$} ] ]
                ]
            ]
        ]
    }

    The semantics of this tree is exactly the desired $\lambda$-term (after reduction):
    \[ and \app germanCity \app (within \app twentykm \app cologne) \]
    And for the other query respectively:
    \[ and \app germanCity \app (withinNorth \app twentykm \app cologne) \]

    Without subtyping, it would be difficult to discern the two categories
    $NorthCheck$ and $DistCheck$ without resorting to things like passing a
    dummy argument whose type encodes whether we are in one of the cases or in the other.
    Note that this cannot be done by simply using a type variable in the place
    of $DistCheck$ and $NorthCheck$ because of the different semantics
    (the categories $DistCheck$ and $NorthCheck$
    combine with different contexts).

    This example may seem strange to some (why would
    we want $less \sq than$ to know about $within$ and $withinNorth$?
    Perhaps it would be better to pair them with $from$?) but in reality the
    other way round is also strange --- these semantics come from $less \sq than$
    and $from$ \emph{combined}. This violation of compositionality arises
    from the natural language itself, and subtyping is a useful ``hack'' to
    deal with it: we tag our type with additional information to denote that
    we are looking for something specific, and when we find it we can unwrap the
    type.
  \end{example}
  \fixednote{не става ясно: кои са they и кои са различните $\lambda$-термове?}
\end{document}
