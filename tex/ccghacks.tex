\documentclass[main.tex]{subfiles}
\begin{document}

\subsection{\hsout{Hacks} Extensions of the CCG formalism}

The CCG formalism described in \ref{sec:ccg} is too weak to represent most
phenomena in natural languages \cite{steedman}. Thus, several extensions have
been developed to deal with them.

\subsubsection{Rule restrictions or modalities}
The only restriction on rules within the pure CCG formalism is the limit to
the number of arguments ($n$). For natural languages, this causes overgeneration.

The simplest method to alleviate this is to allow arbitrary rule restrictions
depending, essentially creating a new formalism for each new target language.
Another method is the so called \emph{slash modalities}, which have been shown
\cite{modal}
to be equivalent to rule restrictions in terms of expressive power, and will
be used here.

\begin{defn}
    The elements of the set
    $\mathcal{M} = \{ \modstar, \modr, \modx, \moddot \}$ are called
    \emph{slash modalities}.

    Slash modalities make up the following lattice:

    \begin{center}
        \begin{tikzpicture}
            \node[main node] (1) {$\modstar$};
            \node[main node] (2) [below left = 1cm and 1.5cm of 1]  {$\modr$};
            \node[main node] (3) [below right = 1cm and 1.5cm of 1] {$\modx$};
            \node[main node] (4) [below = 2cm of 1] {$\moddot$};

            \path[draw,thick]
            (1) edge node {} (2)
            (1) edge node {} (3)
            (2) edge node {} (4)
            (3) edge node {} (4);
        \end{tikzpicture}
    \end{center}
\end{defn}

\begin{defn}
    We extend the definition of categorial closure to use modalities:
    \begin{enumerate}
        \item \label{cmod:atomic} $A \in \tau \Rightarrow A \in C(\tau)$
        \item \label{cmod:right}  $X, Y \in C(\tau), M \in \mathcal{M} \Rightarrow \lp X \rc_M Y \rp \in C(\tau)$
        \item \label{cmod:left}   $X, Y \in C(\tau), M \in \mathcal{M} \Rightarrow \lp X \lc_M Y \rp \in C(\tau)$
    \end{enumerate}
\end{defn}

Now, we restrict derivations according to modalities (in all cases, $m \leq n$):
\begin{itemize}
    \item "Application"
        \[ \lb X \rb \rightarrow \lb X \rc_\modstar Y \rb \lb Y \rb \]
        \[ \lb X \rb \rightarrow \lb Y \rb \lb X \lc_\modstar Y \rb \]
    \item "Harmonic composition"
        \[ \lb X \rc_\modr Z_1 \mc_2 Z_2 ... \mc_m Z_m \rb \rightarrow \lb X \rc_\modr Y \rb \lb Y \rc_\modr Z_1 \mc_2 Z_2 ... \mc_m Z_m \rb \]
        \[ \lb X \lc_\modr Z_1 \mc_2 Z_2 ... \mc_m Z_m \rb \rightarrow \lb Y \lc_\modr Z_1 \mc_2 Z_2 ... \mc_m Z_m \rb \lb X \lc_\modr Y \rb \]
    \item "Crossing composition"
        \[ \lb X \lc_\modx Z_1 \mc_2 Z_2 ... \mc_m Z_m \rb \rightarrow \lb X \rc_\modx Y \rb \lb Y \lc_\modx Z_1 \mc_2 Z_2 ... \mc_m Z_m \rb \]
        \[ \lb X \rc_\modx Z_1 \mc_2 Z_2 ... \mc_m Z_m \rb \rightarrow \lb Y \rc_\modx Z_1 \mc_2 Z_2 ... \mc_m Z_m \rb \lb X \lc_\modx Y \rb \]
\end{itemize}

In the above templates, modalities on the right-hand side stand for any
modality that is less-than or equal (according to the modality lattice)
to the one specified. The modalities on the left-hand side are produced by
unifying (lowest common ancestor) the modalities on the right.

\example{
    For example, here are a few valid derivations:
    \[ \lb X \rb \rightarrow \lb X \rc_\modr Y \rb \lb Y \rb \]
    \[ \lb X \rb \rightarrow \lb X \rc_\modx Y \rb \lb Y \rb \]
    \[ \lb X \rb \rightarrow \lb X \rc_\moddot Y \rb \lb Y \rb \]
    \[ \lb X \rc_\moddot Z_1 \mc_2 Z_2 ... \mc_m Z_m \rb \rightarrow \lb X \rc_\modr Y \rb \lb Y \rc_\moddot Z_1 \mc_2 Z_2 ... \mc_m Z_m \rb \]
    \[ \lb X \rc_\moddot Z_1 \mc_2 Z_2 ... \mc_m Z_m \rb \rightarrow \lb Y \rc_\modx Z_1 \mc_2 Z_2 ... \mc_m Z_m \rb \lb X \lc_\moddot Y \rb \]
}

This way, one can restrict the ways in which a category would
combine by choosing appropriate slash modalities.

When using slash modalities, the bare slashes ($\lc$ and $\rc$) will actually
mean $\lc_\moddot$ and $\rc_\moddot$, being the most generic (they
can combine with any other type of slash).

\subsubsection{Categorial variables}
This is a very powerful extension which adds simple unification based on variables
to categorial derivation. To define it formally, it is convenient to assume
the existence of an infinitely countable set $\mathbb{V}$ whose elements
we call \emph{variables}\footnote{That set is magical in the sense that it is
    disjoint with every set we feel convenient that it be disjoint with.}
and introduce variable substitution over categories.

\begin{defn}
    We extend the definition of categorial closure to include variables:
    \begin{enumerate}
        \item \label{cvar:atomic} $A \in \tau \Rightarrow A \in C(\tau)$
        \item \label{cvar:right}  $X, Y \in C(\tau) \Rightarrow \lp X \rc Y \rp \in C(\tau)$
        \item \label{cvar:left}   $X, Y \in C(\tau) \Rightarrow \lp X \lc Y \rp \in C(\tau)$
        \item \label{cvar:var}    $\alpha \in \mathbb{V} \Rightarrow \alpha \in C(\tau)$
    \end{enumerate}
\end{defn}

For convenience, we will now define a function which gives us all variables
used in a category.
\begin{defn}
    Let $fv: C(\tau) \rightarrow 2^{\mathbb{V}}$ be defined as:
    \[
        fv(Z) =
        \begin{cases*}
            \varnothing, & $Z \in \tau$ \\
            fv(X) \cup fv(Y), & $Z = \lp X \rc Y \rp$ \\
            fv(X) \cup fv(Y), & $Z = \lp X \lc Y \rp$ \\
            \{ \alpha \}, & $Z = \alpha \in \mathbb{V}$ \\
        \end{cases*}
    \]
\end{defn}
\begin{defn}
    Variable substitution over categories is defined as
    \[
        \subst{Z}{\alpha}{W} =
        \begin{cases*}
            Z, & $Z \in \tau$ \\
            \lp \subst{X}{\alpha}{W} \rc \subst{Y}{\alpha}{W} \rp, & $Z = \lp X \rc Y \rp$ \\
            \lp \subst{X}{\alpha}{W} \lc \subst{Y}{\alpha}{W} \rp, & $Z = \lp X \lc Y \rp$ \\
            \beta, & $Z = \beta \in \mathbb{V}$ \\
            W, & $Z = \alpha$ \\
        \end{cases*}
    \]
    for $Z \in C(\tau), W \in C(\tau), \alpha \in \mathbb{V}$
\end{defn}

Having extended the notion of categorial closure to include categorial variables,
we also add a family of derivation rules to deal with variables, namely:
\centree{.{$\subst{X}{\alpha}{Y}$} \edge[very thick]; {$X$} }
for any $X \in C(\tau), Y \in C(\tau), \alpha \in \mathbb{V}$.

To define semantics for derivations with categorial variables, we need to extend
our $\lambda$-calculus to include type-variables and type-variable substitution.
For a more detailed explanation, see \cref{lambda:typevars}.

\begin{center}
    \begin{tikzpicture}[node distance=1mm]
        \Tree[
            .\node(outtop){$\subst{X}{\alpha}{Y}$};
                \edge[very thick];
                [ .\node(intop){$X$};
                    \edge[roof];
                    [ .\node(inbot){$\alpha$}; ]
                ]
        ]
        \node[boxcola,fit=(intop)(inbot)](inbox){};
        \node[right=of inbox,text=\boxtexta](inlabel){$T'$};

        \node[boxcolb,fit=(outtop)(inbot)(inbox)(inlabel)](outbox){};
        \node[right=of outbox,text=\boxtextb](outlabel){$T$};
    \end{tikzpicture}
\end{center}

\[ sem_{\psi}(T) = sem_{\psi}(T')) \]

This essentially allows us to generate any category and substitute it in place
of variables during derivation.

When parsing, we can observe that after processing all leaf nodes,
there's no way to generate new variables while building the parse tree
bottom-up. Thus, it is sufficient to eliminate variables by unification.

Namely, in the $CYK'$ algorithm we can replace the rules (see \ref{cyk:rules})
by the following:

\begin{enumerate}
    \item If $X \in f(w_i)$, then $(X, i, i) \in P$
    \item If $(X \rc Y, i, p) \in P, (Y \mc_1 Z_1 \mc_2 Z_2 ... \mc_m Z_m, p + 1, j) \in P$,
        then $(X \mc_1 Z_1 \mc_2 Z_2 ... \mc_m Z_m, i, j) \in P$
    \item If $(X \lc Y, p + 1, j) \in P, (Y \mc_1 Z_1 \mc_2 Z_2 ... \mc_m Z_m, i, p) \in P$,
        then $(X \mc_1 Z_1 \mc_2 Z_2 ... \mc_m Z_m, i, j) \in P$
\end{enumerate}

\subsubsection{Coordination}
\emph{Coordination} refers to a technique which allows conjunctions (such as
$and$, $or$ etc.) to combine identical categories on both sides.

This can be emulated completely by using categorial variables and setting
$f(and) = \{ \alpha \lc \alpha \rc \alpha \}$\footnote{
    In practice, one would need to set
    $f(and) = \{ X \lc X \rc X, \lp X \mc \alpha \rp \lc \lp X \mc \alpha \rp \rc \lp X \mc \alpha \rp, ...\}$
    in order to properly define semantics.
} and similar.

\subsubsection{Type-raising}
This extension adds a family of unary rules for constructing derivations \cite[sec.~5.3.1]{nts},
parametrised for any slash modality $i$:
\begin{center}
    \tree{.{$\alpha \rc_i \lp \alpha \lc_i X \rp$} \edge[very thick]; {$X$} }
        ( Forward type-raising )
    \tree{.{$\alpha \lc_i \lp \alpha \rc_i X \rp$} \edge[very thick]; {$X$} }
        ( Backward type-raising )
\end{center}

The semantics are defined as follows:
\begin{center}
    \begin{tikzpicture}[node distance=1mm]
        \Tree[
            .\node(outtop){$\alpha \rc_i \lp \alpha \lc_i X \rp$};
                \edge[very thick];
                [ .\node(intop){$X$};
                    \edge[roof];
                    [ .\node(inbot){$\alpha$}; ]
                ]
        ]
        \node[boxcola,fit=(intop)(inbot)](inbox){};
        \node[right=of inbox,text=\boxtexta](inlabel){$T'$};

        \node[boxcolb,fit=(outtop)(inbot)(inbox)(inlabel)](outbox){};
        \node[right=of outbox,text=\boxtextb](outlabel){$T$};
    \end{tikzpicture}
    \hspace{2cm}
    \begin{tikzpicture}[node distance=1mm]
        \Tree[
            .\node(outtop){$\alpha \lc_i \lp \alpha \rc_i X \rp$};
                \edge[very thick];
                [ .\node(intop){$X$};
                    \edge[roof];
                    [ .\node(inbot){$\alpha$}; ]
                ]
        ]
        \node[boxcola,fit=(intop)(inbot)](inbox){};
        \node[right=of inbox,text=\boxtexta](inlabel){$T'$};

        \node[boxcolb,fit=(outtop)(inbot)(inbox)(inlabel)](outbox){};
        \node[right=of outbox,text=\boxtextb](outlabel){$T$};
    \end{tikzpicture}
\end{center}

\[ sem_{\psi}(T) = \lambda f {:} ( X \tot \alpha ) \abstr f (sem_{\psi}(T')) \]

Since considering type-raising at any level would make parsing much slower,
it is usually only considered on categories which appear in leaves.

\example{
    Consider the following simple grammar:
    \begin{align*}
        f(cities) &= \{ Set \} \\
        f(villages) &= \{ Set \} \\
        f(Finland) &= \{ Set \} \\
        f(and) &= \{ \alpha \lc \alpha \rc \alpha \} \\
        f(in) &= \{ Set \lc Set \rc Set \} \\
        f(near) &= \{ Set \lc Set \rc Set \} \\
    \end{align*}

    It permits derivations as such:

    \centree{
        .{$Set$}
            \edge[very thick];
            [ .{$Set$}
                [ .{$Set$}
                    [ .{$cities$} ]
                ]
                [ .{$Set \lc Set$}
                    [ .{$Set \lc Set \rc Set$}
                        [ .{$\alpha \lc \alpha \rc \alpha$} [ .{$and$} ] ]
                    ]
                    [ .{$Set$}
                        [ .{$villages$} ]
                    ]
                ]
            ]
            [ .{$Set \lc Set$}
                [ .{$Set \lc Set \rc Set$}
                    [ .{$in$} ]
                ]
                [ .{$Set$}
                    [ .{$Finland$} ]
                ]
            ]
    }

    If we would like to be able to accept queries like "cities in and villages
    near Finland", we could permit type-raising:

    \scaledtree{0.7}{
        .{$Set$}
        [ .{$Set \rc Set$}
            [ .{$ Set \rc Set$}
                [ .{$Set \rc \lp Set \lc Set \rp$}
                    [ .{$\alpha \rc \lp \alpha \lc Set \rp$} [ .{$Set$} [ .{$cities$} ] ] ]
                ]
                [ .{$Set \lc Set \rc Set$} [ .{$in$} ]
                ]
            ]
            [ .{$ \lp Set \rc Set \rp \lc \lp Set \rc Set \rp$}
                [ .{$\lp Set \rc Set \rp \lc \lp Set \rc Set \rp \rc \lp Set \rc Set \rp$}
                    [ .{$\alpha \lc \alpha \rc \alpha$} [ .{$and$} ] ]
                ]
                    [ .{$Set \rc Set$}
                    [ .{$Set \rc \lp Set \lc Set \rp$}
                        [ .{$\alpha \rc \lp \alpha \lc Set \rp$} [ .{$Set$} [ .{$villages$} ] ] ]
                    ]
                    [ .{$Set \lc Set \rc Set$} [ .{$near$} ]
                    ]
                ]
            ]
        ]
        [ .{$ Set$} [ .{$Finland$} ] ]
    }
}

\subsubsection{Unification of categories}
To reduce the size and complexity of grammars, it is convenient to regard
not only identical categories as equal, but rather use an \emph{unification}
operation which can judge whether two categories unify and what their common
unifier should be.

One method is to explicitly state the unification rules in the grammar.
It is also useful to make categories complex objects and infer unification
based on their structure, as will be done in the next two sections.

\subsubsection{Unification by category features}
Some models allow each simple category to be extended with a set of \emph{features}
which refine its scope, and then define rules for unification of said features.

For example, we could assume a global set of features $Feat$ to be the
set of finite functions $f: [a-zA-Z\_]^* \rightarrow [a-zA-Z\_]^*$
and instead of using $C(\tau)$ directly, we use $C(\tau \times Feat)$.
For brevity, categories like $(V, \{ (tense, past), (person, first) \})$
are written as $X[tense=past, person=first]$.

A common unification scheme is to do an union of the two feature functions,
and see if the result is also a function - this determines whether the
categories actually unify.

This can be further extended by concepts such as \emph{feature variables}
(which allows the use of variables within feature expressions and
performs variable substitution during unification) and embedding regular expressions.

\subsubsection{Unification by subtyping}
If the grammar will be used for generating typed $\lambda$-terms which use
a subtyping system like the one described in \autoref{sec:lambda}, the subtyping
relation can be used to unify categories.

One approach is to make any two categories which are comparable by the subtyping
relation unify to the lower of the two: essentially using the $\meet$-operator
as a unification operator. This, however, can result in terms with downcasts,
which might not be the desired result.

If the desired semantics is to allow functions to only accept subtypes of
their argument, some extra care needs to be taken, namely
\[ \lb X \mc_1 Z_1 \mc_2 Z_2 ... \mc_m Z_m \rb \rightarrow \lb X \rc Y_1 \rb \lb Y_2 \mc_1 Z_1 \mc_2 Z_2 ... \mc_m Z_m \rb \]
\[ \lb X \mc_1 Z_1 \mc_2 Z_2 ... \mc_m Z_m \rb \rightarrow \lb Y_2 \mc_1 Z_1 \mc_2 Z_2 ... \mc_m Z_m \rb \lb X \lc Y_1 \rb \]
only if $Y_2 \less Y_1$.
\end{document}
