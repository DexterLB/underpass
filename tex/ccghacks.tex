\documentclass[main.tex]{subfiles}
\begin{document}

\subsection{\hsout{Hacks} Extensions of the CCG formalism}

The CCG formalism described in \ref{sec:ccg} is too weak to represent most
phenomena in natural languages \cite{steedman}. Thus, several extensions have
been developed to deal with them.

\subsubsection{Rule restrictions or modalities}
The only restriction on rules within the pure CCG formalism is the limit to
the number of arguments ($n$). For natural languages, this causes overgeneration.

The simplest method to alleviate this is to allow arbitrary rule restrictions
depending, essentially creating a new formalism for each new target language.
Another method is the so called \emph{slash modalities}, which have been shown
\cite{modal}
to be equivalent to rule restrictions in terms of expressive power, and will
be used here.

\begin{defn}
    The elements of the set
    $\mathcal{M} = \{ \modstar, \modr, \modx, \moddot \}$ are called
    \emph{slash modalities}.

    Slash modalities make up the following lattice:

    \begin{center}
        \begin{tikzpicture}
            \node[main node] (1) {$\modstar$};
            \node[main node] (2) [below left = 1cm and 1.5cm of 1]  {$\modr$};
            \node[main node] (3) [below right = 1cm and 1.5cm of 1] {$\modx$};
            \node[main node] (4) [below = 2cm of 1] {$\moddot$};

            \path[draw,thick]
            (1) edge node {} (2)
            (1) edge node {} (3)
            (2) edge node {} (4)
            (3) edge node {} (4);
        \end{tikzpicture}
    \end{center}
\end{defn}

\begin{defn}
    We extend the definition of categorial closure to use modalities:
    \begin{enumerate}
        \item \label{cmod:atomic} $A \in \tau \Rightarrow A \in C(\tau)$
        \item \label{cmod:right}  $X, Y \in C(\tau), M \in \mathcal{M} \Rightarrow \lp X \rc_M Y \rp \in C(\tau)$
        \item \label{cmod:left}   $X, Y \in C(\tau), M \in \mathcal{M} \Rightarrow \lp X \lc_M Y \rp \in C(\tau)$
    \end{enumerate}
\end{defn}

Now, we restrict derivations according to modalities (in all cases, $m \leq n$):
\begin{itemize}
    \item "Application"
        \[ \lb X \rb \rightarrow \lb X \rc_\modstar Y \rb \lb Y \rb \]
        \[ \lb X \rb \rightarrow \lb Y \rb \lb X \lc_\modstar Y \rb \]
    \item "Harmonic composition"
        \[ \lb X \rc_\modr Z_1 \mc_2 Z_2 ... \mc_m Z_m \rb \rightarrow \lb X \rc_\modr Y \rb \lb Y \rc_\modr Z_1 \mc_2 Z_2 ... \mc_m Z_m \rb \]
        \[ \lb X \lc_\modr Z_1 \mc_2 Z_2 ... \mc_m Z_m \rb \rightarrow \lb Y \lc_\modr Z_1 \mc_2 Z_2 ... \mc_m Z_m \rb \lb X \lc_\modr Y \rb \]
    \item "Crossing composition"
        \[ \lb X \lc_\modx Z_1 \mc_2 Z_2 ... \mc_m Z_m \rb \rightarrow \lb X \rc_\modx Y \rb \lb Y \lc_\modx Z_1 \mc_2 Z_2 ... \mc_m Z_m \rb \]
        \[ \lb X \rc_\modx Z_1 \mc_2 Z_2 ... \mc_m Z_m \rb \rightarrow \lb Y \rc_\modx Z_1 \mc_2 Z_2 ... \mc_m Z_m \rb \lb X \lc_\modx Y \rb \]
\end{itemize}

In the above templates, modalities on the right-hand side stand for any
modality that is less-than or equal (according to the modality lattice)
to the one specified. The modalities on the left-hand side are produced by
unifying (lowest common ancestor) the modalities on the right.

\example{
    For example, here are a few valid derivations:
    \[ \lb X \rb \rightarrow \lb X \rc_\modr Y \rb \lb Y \rb \]
    \[ \lb X \rb \rightarrow \lb X \rc_\modx Y \rb \lb Y \rb \]
    \[ \lb X \rb \rightarrow \lb X \rc_\moddot Y \rb \lb Y \rb \]
    \[ \lb X \rc_\moddot Z_1 \mc_2 Z_2 ... \mc_m Z_m \rb \rightarrow \lb X \rc_\modr Y \rb \lb Y \rc_\moddot Z_1 \mc_2 Z_2 ... \mc_m Z_m \rb \]
    \[ \lb X \rc_\moddot Z_1 \mc_2 Z_2 ... \mc_m Z_m \rb \rightarrow \lb Y \rc_\modx Z_1 \mc_2 Z_2 ... \mc_m Z_m \rb \lb X \lc_\moddot Y \rb \]
}

\subsubsection{Categorial variables}
This is a very powerful extension which adds simple unification based on variables
to categorial derivation. To define it formally, it is convenient to assume
the existence of an infinitely countable set $\mathbb{V}$ whose elements
we call \emph{variables}\footnote{That set is magical in the sense that it is
    disjoint with every set we feel convenient that it be disjoint with.}
and introduce variable substitution over categories.

\begin{defn}
    We extend the definition of categorial closure to include variables:
    \begin{enumerate}
        \item \label{cvar:atomic} $A \in \tau \Rightarrow A \in C(\tau)$
        \item \label{cvar:right}  $X, Y \in C(\tau) \Rightarrow \lp X \rc Y \rp \in C(\tau)$
        \item \label{cvar:left}   $X, Y \in C(\tau) \Rightarrow \lp X \lc Y \rp \in C(\tau)$
        \item \label{cvar:var}    $\alpha \in \mathbb{V} \Rightarrow \alpha \in C(\tau)$
    \end{enumerate}
\end{defn}

\begin{defn}
    The function $fv: C(\tau) \rightarrow 2^{\mathbb{V}}$ gives us the set of
    all variables used in a category:
    \[
        fv(Z) =
        \begin{cases*}
            \varnothing, & $Z \in \tau$ \\
            fv(X) \cup fv(Y), & $Z = \lp X \rc Y \rp$ \\
            fv(X) \cup fv(Y), & $Z = \lp X \lc Y \rp$ \\
            \{ \alpha \}, & $Z = \alpha \in \mathbb{V}$ \\
        \end{cases*}
    \]
\end{defn}
\begin{defn}
    Variable substitution over categories is defined as
    \[
        \subst{Z}{\alpha}{W} =
        \begin{cases*}
            Z, & $Z \in \tau$ \\
            \lp \subst{X}{\alpha}{W} \rc \subst{Y}{\alpha}{W} \rp, & $Z = \lp X \rc Y \rp$ \\
            \lp \subst{X}{\alpha}{W} \lc \subst{Y}{\alpha}{W} \rp, & $Z = \lp X \lc Y \rp$ \\
            \beta, & $Z = \beta \in \mathbb{V}$ \\
            W, & $Z = \alpha$ \\
        \end{cases*}
    \]
    for $Z \in C(\tau), W \in C(\tau), \alpha \in \mathbb{V}$
\end{defn}

Having extended the notion of categorial closure to include categorial variables,
we also add a family of derivation rules to deal with variables, namely:
\begin{center}
    \tree{.{$\lb \subst{X}{\alpha}{Y} \rb$} \edge[very thick]; {$\lb X \rb$} }
\end{center}
for any $X \in C(\tau), Y \in C(\tau), \alpha \in \mathbb{V}$.

This essentially allows us to generate any category and substitute it in place
of variables during derivation.

When parsing, we can observe that after processing all leaf nodes,
there's no way to generate new variables while building the parse tree
bottom-up. Thus, it is sufficient to eliminate variables by unification.

Namely, in the $CYK'$ algorithm we can replace the rules (see \ref{cyk:rules})
by the following:

\begin{enumerate}
    \item If $X \in f(w_i)$, then $(X, i, i) \in P$
    \item If $(X \rc Y, i, p) \in P, (Y \mc_1 Z_1 \mc_2 Z_2 ... \mc_m Z_m, p + 1, j) \in P$,
        then $(X \mc_1 Z_1 \mc_2 Z_2 ... \mc_m Z_m, i, j) \in P$
    \item If $(X \lc Y, p + 1, j) \in P, (Y \mc_1 Z_1 \mc_2 Z_2 ... \mc_m Z_m, i, p) \in P$,
        then $(X \mc_1 Z_1 \mc_2 Z_2 ... \mc_m Z_m, i, j) \in P$
\end{enumerate}

\subsubsection{Coordination}
\emph{Coordination} refers to a technique which allows conjunctions (such as
$and$, $or$ etc.) to combine identical categories on both sides.

This can be emulated completely by using categorial variables and setting
$f(and) = \{ \alpha \lc \alpha \rc \alpha \}$\footnote{
    In practice, one would need to set
    $f(and) = \{ X \lc X \rc X, \lp X \mc \alpha \rp \lc \lp X \mc \alpha \rp \rc \lp X \mc \alpha \rp, ...\}$
    in order to properly define semantics.
} and similar.

\subsubsection{Type-raising}
This extension adds two unary rules for constructing derivations: \cite[sec.~5.3.1]{nts}
\begin{center}
    \tree{.{$\lb \alpha \rc \lp \alpha \lc X \rp \rb$} \edge[very thick]; {$\lb X \rb$} }
        ( Forward type-raising )
    \tree{.{$\lb \alpha \lc \lp \alpha \rc X \rp \rb$} \edge[very thick]; {$\lb X \rb$} }
        ( Backward type-raising )
\end{center}

Since considering type-raising at any level would make parsing much slower,
it is usually only considered on categories which appear in leaves.

\example{
    Consider the following simple grammar:
    \begin{align*}
        f(cities) &= \{ Set \} \\
        f(villages) &= \{ Set \} \\
        f(Finland) &= \{ Set \} \\
        f(and) &= \{ \alpha \lc \alpha \rc \alpha \} \\
        f(in) &= \{ Set \lc Set \rc Set \} \\
        f(near) &= \{ Set \lc Set \rc Set \} \\
    \end{align*}

    It permits derivations as such:

    \centree{
        .{$\lb Set \rb$}
            \edge[very thick];
            [ .{$\lb Set \rb$}
                [ .{$\lb Set \rb$}
                    [ .{$cities$} ]
                ]
                [ .{$\lb Set \lc Set \rb$}
                    [ .{$\lb Set \lc Set \rc Set \rb$}
                        [ .{$\lb \alpha \lc \alpha \rc \alpha \rb$} [ .{$and$} ] ]
                    ]
                    [ .{$\lb Set \rb$}
                        [ .{$villages$} ]
                    ]
                ]
            ]
            [ .{$\lb Set \lc Set \rb$}
                [ .{$\lb Set \lc Set \rc Set \rb$}
                    [ .{$in$} ]
                ]
                [ .{$\lb Set \rb$}
                    [ .{$Finland$} ]
                ]
            ]
    }

    If we would like to be able to accept queries like "cities in and villages
    near Finland", we could permit type-raising:

    \scaledtree{0.7}{
        .{$\lb Set \rb$}
        [ .{$\lb Set \rc Set \rb$}
            [ .{$ \lb Set \rc Set \rb$}
                [ .{$\lb Set \rc \lp Set \lc Set \rp \rb$}
                    [ .{$\lb \alpha \rc \lp \alpha \lc Set \rp \rb$} [ .{$\lb Set \rb$} [ .{$cities$} ] ] ]
                ]
                [ .{$\lb Set \lc Set \rc Set \rb$} [ .{$in$} ]
                ]
            ]
            [ .{$ \lb \lp Set \rc Set \rp \lc \lp Set \rc Set \rp \rb$}
                [ .{$\lb \lp Set \rc Set \rp \lc \lp Set \rc Set \rp \rc \lp Set \rc Set \rp \rb$}
                    [ .{$\lb \alpha \lc \alpha \rc \alpha \rb$} [ .{$and$} ] ]
                ]
                    [ .{$\lb Set \rc Set \rb$}
                    [ .{$\lb Set \rc \lp Set \lc Set \rp \rb$}
                        [ .{$\lb \alpha \rc \lp \alpha \lc Set \rp \rb$} [ .{$\lb Set \rb$} [ .{$villages$} ] ] ]
                    ]
                    [ .{$\lb Set \lc Set \rc Set \rb$} [ .{$near$} ]
                    ]
                ]
            ]
        ]
        [ .{$ \lb Set \rb$} [ .{$Finland$} ] ]
    }
}

\end{document}
